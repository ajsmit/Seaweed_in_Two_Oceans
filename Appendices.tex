\documentclass[10pt,A4,]{article}
% \RequirePackage[l2tabu, orthodox]{nag}
\usepackage[a4paper,text={16.5cm,25.2cm},centering,margin=2.6cm]{geometry}
% \usepackage[left=1.0in,top=1.0in,right=1.0in,bottom=1.0in]{geometry}
\newcommand*{\authorfont}{\fontfamily{phv}\selectfont}
\usepackage{hyperref,amsmath,amssymb,bm,url,enumitem,dcolumn,upquote,framed,alltt,textgreek,xfrac,fixltx2e}
\usepackage[australian]{babel}
\usepackage[compact,small]{titlesec}
\setlength{\parskip}{1.2ex}
\setlength{\parindent}{0em}

\def\tightlist{}

\usepackage{ifxetex}
\ifxetex
  \usepackage{fontspec}
  \defaultfontfeatures{Ligatures=TeX} % To support LaTeX quoting style
 \defaultfontfeatures{Ligatures=TeX,Numbers={OldStyle}}
 \setmainfont{Minion Pro}
 \setsansfont[Scale=MatchLowercase]{Myriad Pro}
 \setmonofont[Scale=MatchLowercase]{Ubuntu Mono}
\else
  \usepackage[T1]{fontenc}
  \usepackage[utf8]{inputenc}
  \usepackage{lmodern}
  % \usepackage[full]{textcomp} % directly use the degree (and some other) symbol
\fi

% place after fonts; even better typesetting for improved readability:
\graphicspath{ {figure/} }
\usepackage{tabularx} % for 'tabularx' environment and 'X' column type
\usepackage{ragged2e}  % for '\RaggedRight' macro (allows hyphenation)
\usepackage{siunitx}
    \sisetup{%
        detect-mode,
        group-digits            = false,
        input-symbols           = ( ) [ ] - + < > *,
        table-align-text-post   = false,
        round-mode              = places,
        round-precision         = 3
        }
% \usepackage[font={small, sf}, labelfont=bf]{caption} % tweaking the captions
\usepackage[font={small}, labelfont=bf]{caption} % tweaking the captions
\usepackage[color=yellow, textsize=tiny]{todonotes}
\frenchspacing%
% \usepackage[kerning=false,protrusion=true,expansion=true]{microtype}

\usepackage{abstract}
\renewcommand{\abstractname}{} % clear the title
\renewcommand{\absnamepos}{empty} % originally center

\renewenvironment{abstract}
{{%
\setlength{\leftmargin}{0mm}
\setlength{\rightmargin}{\leftmargin}%
}%
\relax}
{\endlist}

\makeatletter
\def\@maketitle{%
\newpage
%  \null
%  \vskip 2em%
%  \begin{center}%
\let \footnote \thanks
 {\fontsize{18}{20}\selectfont\raggedright  \setlength{\parindent}{0pt} \@title \par}%
}
%\fi
\makeatother


\setcounter{secnumdepth}{0}



\usepackage{color}
\usepackage{fancyvrb}
\newcommand{\VerbBar}{|}
\newcommand{\VERB}{\Verb[commandchars=\\\{\}]}
\DefineVerbatimEnvironment{Highlighting}{Verbatim}{commandchars=\\\{\}}
% Add ',fontsize=\small' for more characters per line
\usepackage{framed}
\definecolor{shadecolor}{RGB}{248,248,248}
\newenvironment{Shaded}{\begin{snugshade}}{\end{snugshade}}
\newcommand{\KeywordTok}[1]{\textcolor[rgb]{0.13,0.29,0.53}{\textbf{#1}}}
\newcommand{\DataTypeTok}[1]{\textcolor[rgb]{0.13,0.29,0.53}{#1}}
\newcommand{\DecValTok}[1]{\textcolor[rgb]{0.00,0.00,0.81}{#1}}
\newcommand{\BaseNTok}[1]{\textcolor[rgb]{0.00,0.00,0.81}{#1}}
\newcommand{\FloatTok}[1]{\textcolor[rgb]{0.00,0.00,0.81}{#1}}
\newcommand{\ConstantTok}[1]{\textcolor[rgb]{0.00,0.00,0.00}{#1}}
\newcommand{\CharTok}[1]{\textcolor[rgb]{0.31,0.60,0.02}{#1}}
\newcommand{\SpecialCharTok}[1]{\textcolor[rgb]{0.00,0.00,0.00}{#1}}
\newcommand{\StringTok}[1]{\textcolor[rgb]{0.31,0.60,0.02}{#1}}
\newcommand{\VerbatimStringTok}[1]{\textcolor[rgb]{0.31,0.60,0.02}{#1}}
\newcommand{\SpecialStringTok}[1]{\textcolor[rgb]{0.31,0.60,0.02}{#1}}
\newcommand{\ImportTok}[1]{#1}
\newcommand{\CommentTok}[1]{\textcolor[rgb]{0.56,0.35,0.01}{\textit{#1}}}
\newcommand{\DocumentationTok}[1]{\textcolor[rgb]{0.56,0.35,0.01}{\textbf{\textit{#1}}}}
\newcommand{\AnnotationTok}[1]{\textcolor[rgb]{0.56,0.35,0.01}{\textbf{\textit{#1}}}}
\newcommand{\CommentVarTok}[1]{\textcolor[rgb]{0.56,0.35,0.01}{\textbf{\textit{#1}}}}
\newcommand{\OtherTok}[1]{\textcolor[rgb]{0.56,0.35,0.01}{#1}}
\newcommand{\FunctionTok}[1]{\textcolor[rgb]{0.00,0.00,0.00}{#1}}
\newcommand{\VariableTok}[1]{\textcolor[rgb]{0.00,0.00,0.00}{#1}}
\newcommand{\ControlFlowTok}[1]{\textcolor[rgb]{0.13,0.29,0.53}{\textbf{#1}}}
\newcommand{\OperatorTok}[1]{\textcolor[rgb]{0.81,0.36,0.00}{\textbf{#1}}}
\newcommand{\BuiltInTok}[1]{#1}
\newcommand{\ExtensionTok}[1]{#1}
\newcommand{\PreprocessorTok}[1]{\textcolor[rgb]{0.56,0.35,0.01}{\textit{#1}}}
\newcommand{\AttributeTok}[1]{\textcolor[rgb]{0.77,0.63,0.00}{#1}}
\newcommand{\RegionMarkerTok}[1]{#1}
\newcommand{\InformationTok}[1]{\textcolor[rgb]{0.56,0.35,0.01}{\textbf{\textit{#1}}}}
\newcommand{\WarningTok}[1]{\textcolor[rgb]{0.56,0.35,0.01}{\textbf{\textit{#1}}}}
\newcommand{\AlertTok}[1]{\textcolor[rgb]{0.94,0.16,0.16}{#1}}
\newcommand{\ErrorTok}[1]{\textcolor[rgb]{0.64,0.00,0.00}{\textbf{#1}}}
\newcommand{\NormalTok}[1]{#1}


\usepackage{longtable,booktabs}
\setlength\heavyrulewidth{0.1em}
\setlength\lightrulewidth{0.0625em}


\title{Seaweeds in Two Oceans: Beta-diversity (Appendices)  }

\author{\Large AJ Smit\vspace{0.05in} \newline\normalsize\emph{University of the Western Cape}   \and \Large JJ Bolton\vspace{0.05in} \newline\normalsize\emph{University of Cape Town}   \and \Large RJ Anderson\vspace{0.05in} \newline\normalsize\emph{Department of Agriculture, Forestry and Fisheries and University of Cape
Town}  }

\date{}

% PENALTIES
\widowpenalty=1000
\clubpenalty=1000
\doublehyphendemerits=9999 % Almost no consecutive line hyphens
\brokenpenalty=10000 % No broken words across columns/pages
\interfootnotelinepenalty=9999 % Almost never break footnotes

% SECTION, SUBSECETC.TITLES
\usepackage[compact]{titlesec}
\titleformat{\chapter}
  {\normalfont\LARGE\sffamily\bfseries}
  {\thechapter}
  {1em}
  {}
\titleformat{\section}
  {\normalfont\LARGE\sffamily\bfseries}
  {\thesection}
  {1em}
  {}
\titleformat{\subsection}
  {\normalfont\Large\sffamily\bfseries}
  {\thesubsection}
  {1em}
  {}
\titleformat{\subsubsection}
  {\normalfont\large\sffamily\bfseries\slshape}
  {\thesubsubsection}
  {1em}
  {}
% \titlespacing*{<command>}{<left>}{<before-sep>}{<after-sep>}
\titlespacing*{\chapter}
  {0pt}
  {1.2ex plus 1ex minus .2ex}
  {0.5ex plus .1ex minus .1ex}
\titlespacing*{\section}
  {0pt}
  {1.2ex plus 1ex minus .2ex}
  {0.5ex plus .1ex minus .1ex}
\titlespacing*{\subsection}
  {0pt}
  {1.2ex plus 1ex minus .2ex}
  {0.5ex plus .1ex minus .1ex}
\titlespacing*{\subsubsection}
  {0pt}
  {1.2ex plus 1ex minus .2ex}
  {0.5ex plus .1ex minus .1ex}



\newtheorem{hypothesis}{Hypothesis}
\usepackage{setspace}

\makeatletter

\@ifpackageloaded{hyperref}{}{%
\ifxetex
  \usepackage[setpagesize=false, % page size defined by xetex
              unicode=false, % unicode breaks when used with xetex
              xetex]{hyperref}
\else
  \usepackage[colorlinks=true, citecolor=blue, linkcolor=cyan, pdfborder={0 0 0 }, unicode=true]{hyperref} % place after other packages, but before cleveref
\fi
}

\@ifpackageloaded{color}{
    \PassOptionsToPackage{usenames,dvipsnames}{color}
}{%
    \usepackage[usenames,dvipsnames]{color}
}

\usepackage{cleveref} % cleverly cross referencing figures and tables; last package to include
\setcounter{secnumdepth}{2}
\setcounter{tocdepth}{2}

% To use for resetting the numbering of Appendix Tables and Figures:
%\setcounter{table}{0}
%\renewcommand{\thetable}{A\arabic{table}}
%\setcounter{figure}{0}
%\renewcommand{\thefigure}{A\arabic{figure}}

\makeatother
\hypersetup{breaklinks=true,
            bookmarks=true,
            pdfauthor={AJ Smit (University of the Western Cape) and JJ Bolton (University of Cape Town) and RJ Anderson (Department of Agriculture, Forestry and Fisheries and University of Cape
Town)},
             pdfkeywords = {},
            pdftitle={Seaweeds in Two Oceans: Beta-diversity (Appendices)},
            colorlinks=true,
            citecolor=blue,
            urlcolor=blue,
            linkcolor=magenta,
            pdfborder={0 0 0}}
\urlstyle{same}  % don't use monospace font for urls


\begin{document}

% \maketitle

{% \usefont{T1}{pnc}{m}{n}
\setlength{\parindent}{0pt}
\thispagestyle{plain}
{\fontsize{18}{20}\selectfont\raggedright
\maketitle  % title \par
}
{
   \vskip 13.5pt\relax \normalsize\fontsize{11}{12}
\textbf{\authorfont AJ Smit} \hskip 15pt \emph{\small University of the Western Cape}   \par \textbf{\authorfont JJ Bolton} \hskip 15pt \emph{\small University of Cape Town}   \par \textbf{\authorfont RJ Anderson} \hskip 15pt \emph{\small Department of Agriculture, Forestry and Fisheries and University of Cape
Town}   
}
}



\vskip 6.5pt

\noindent 

This file was generated in R using Rmarkdown, with a bit of
\LaTeX~thrown in:

\begin{Shaded}
\begin{Highlighting}[]
\KeywordTok{sessionInfo}\NormalTok{()}
\end{Highlighting}
\end{Shaded}

\begin{verbatim}
## R version 3.4.1 (2017-06-30)
## Platform: x86_64-apple-darwin15.6.0 (64-bit)
## Running under: macOS Sierra 10.12.6
## 
## Matrix products: default
## BLAS: /Library/Frameworks/R.framework/Versions/3.4/Resources/lib/libRblas.0.dylib
## LAPACK: /Library/Frameworks/R.framework/Versions/3.4/Resources/lib/libRlapack.dylib
## 
## locale:
## [1] en_US.UTF-8/en_US.UTF-8/en_US.UTF-8/C/en_US.UTF-8/en_US.UTF-8
## 
## attached base packages:
## [1] stats     graphics  grDevices utils     datasets  methods   base     
## 
## loaded via a namespace (and not attached):
##  [1] compiler_3.4.1  backports_1.1.1 magrittr_1.5    rprojroot_1.2  
##  [5] tools_3.4.1     htmltools_0.3.6 yaml_2.1.14     Rcpp_0.12.13   
##  [9] stringi_1.1.5   rmarkdown_1.6   knitr_1.17      stringr_1.2.0  
## [13] digest_0.6.12   evaluate_0.10.1
\end{verbatim}

\newpage

\section*{\large{Appendix A}}

\begin{tiny}
\begin{center}
\setlength\tabcolsep{4pt}
\begin{longtable}{|r|r|r|l|p{6cm}|}
\caption{The 58 \texttimes{} 50 km sections of the South African coastline, with approximate GPS coordinates, delineation of sections, and some well-known sites in each section. Taken with permission from Bolton and Stegenga (2002).} \\

% This is the header for the first page of the table...
\toprule
Site & Lon & Lat & Limits & Including \\
\midrule
\endfirsthead

% This is the header for the second page of the table...
\toprule
Site & Lon & Lat & Limits & Including \\
\midrule
\endhead

% This is the footer for all pages except the last page of the table...
\midrule
\multicolumn{5}{l}{{Continued on Next Page\ldots}} \\
\endfoot

% This is the footer for the last page of the table...
\bottomrule
\endlastfoot

% Now the data...
1 & 16.72 & -28.98 & Orange River to just south of Holgats River  &  \\
2 & 16.94 & -29.38 & To just south of Wedge Point & Port Nolloth \\
3 & 17.08 & -29.83 & To just south of Melkbos Point & Kleinzee \\
4 & 17.26 & -30.26 & To Swartlintjies River & Skulpfontein Point, Swartlintjies \\
5 & 17.48 & -30.68 & To 10 km north of Groen River  & River Hondeklip Bay, Spoeg River \\
6 & 17.72 & -31.09 & To just north of Brak River & Groen River, Island Point, Blougat \\
7 & 18.00 & -31.46 & To just north of Duiwegat & Voëlklip, Sout River, Blinkwater Bay \\
8 & 18.25 & -31.85 & To just south of Doring Bay & Olifants River, Strandfontein, Doring Bay \\
9 & 18.34 & -32.30 & To just north of Elands Bay & Lambert’s Bay, Lang River \\
10 & 18.20 & -32.72 & To just north of Laaiplek & Elands Bay, Die Vlei, Dwarskersbos \\
11 & 17.85 & -32.83 & To just south of Cape Columbine & Laaiplek, St Helena Bay, Paternoster \\
12 & 18.03 & -33.03 & To just east of Saldahna &  \\
13 & 18.01 & -33.15 & To Postberg & Langebaan Lagoon \\
14 & 18.32 & -33.50 & To just south of Modder River & Yzerfontein, Dassen Is., Grotto Bay \\
15 & 18.47 & -33.91 & To Sea Point & Melkbosstrand, Table Bay, Green Point \\
16 & 18.37 & -34.21 & To just north of Scarborough & Camps Bay, Hout Bay, Kommetjie \\
17 & 18.47 & -34.11 & To just east of Kalk Bay & Scarborough, Cape Point, Fishoek \\
18 & 18.82 & -34.19 & To just south of Gordons Bay & Muizenburg, Strandfontein, Strand \\
19 & 19.07 & -34.35 & To just east of Kleinmond & Rooi Els, Hangklip, Betty’s Bay \\
20 & 19.34 & -34.59 & To just south of Danger Point & Bot River, Sand Bay, Hermanus, Die Kelders \\
21 & 19.66 & -34.79 & To just east of Quoin Point & Danger Point, Pearly Beach, Dyer Island \\
22 & 20.07 & -34.75 & To just east of Struis Bay & Die Mond, Cape Agulhas \\
23 & 20.48 & -34.49 & To just east of Skipskop & Struis Bay, Arniston \\
24 & 20.87 & -34.39 & To just east of Cape Infanta & Koppie Alleen, Cape Infanta, Bree River, Witsand \\
25 & 21.36 & -34.42 & To just east of Grootjongensfontein & Puntjie, Skurwe Bay \\
26 & 21.83 & -34.38 & To just west of Gouritzmond & Stil Bay, Bloukrans, Bull Point \\
27 & 22.12 & -34.16 & To just north of Mossel Bay & Gouritzmond, Vlees Bay, Pinnacle Rock \\
28 & 22.54 & -34.01 & To just west of Victoria Bay & Hartenbos, Klein and Groot Brak rivers, Herolds Bay, Skuinsbank \\
29 & 23.02 & -34.08 & To just west of The Heads, Knysna & Victoria Bay, Wilderness, Platbank, Oesterbank, Walker Bay \\
30 & 23.36 & -34.10 & To Jack’s Point, south of Plettenberg Bay  & The Heads, Neusgate \\
31 & 23.78 & -34.01 & To Elandbos River & Plettenberg Bay, Arch Rock, Die Punt, Blousloep \\
32 & 24.27 & -34.08 & To Skuinsklip & Storms River, Voëlkrans, Skietgate \\
33 & 24.74 & -34.19 & To Thys Point & Aasvoëlklip, Tsitsikamma River, Klipdrif River \\
34 & 25.04 & -33.97 & To just west of Gamtoos River & Cape St Francis, Krom River, Seekoei River, Jeffreys Bay \\
35 & 25.52 & -34.04 & To just east of Sardinia Bay & Van Stadens River, Claasen Point \\
36 & 25.70 & -33.79 & To just east of St George’s Beach & Chelsea Point, Port Elizabeth, Bluewater Bay \\
37 & 26.18 & -33.72 & To just west of Woody Cape & St Croix Is., Sundays River \\
38 & 26.65 & -33.70 & To just west of Kenton-on-Sea & Seal Is., Bird Is., Cape Padrone, Cannon Rocks, Boknes \\
39 & 27.10 & -33.52 & To just east of Kleinemonde & Kasouga, Port Alfred \\
40 & 27.52 & -33.27 & To just east of Keiskamma River & Great Fish River, Madagascar Reef \\
41 & 27.93 & -33.01 & To just east of East London & Kayser’s Beach, Kidd’s Beach, Cove Rock \\
42 & 28.30 & -32.73 & To Haga-Haga & Gonubie, Cintsa River \\
43 & 28.68 & -32.44 & To Qora River & Morgans Bay, Kei Mouth, Nxaxo River, Mazeppa Bay \\
44 & 29.05 & -32.11 & To just east of Xora River & Dwesa, The Haven \\
45 & 29.37 & -31.76 & To Sharks Point & Mncwasa River, Coffee Bay, Hluleka \\
46 & 29.74 & -31.46 & To Mkozi River & Boulder Bay, Port St Johns, Montshe, Ntsubane \\
47 & 30.12 & -31.18 & To Mnyameni River & Cathedral Rock, Lambasi Bay, Wild Coast \\
48 & 30.41 & -30.81 & To just north of St Michaels-on-Sea & Mzamba, Port Edward, Southbroom, Margate \\
49 & 30.68 & -30.41 & To just south of Pennington & Port Shepstone, Mzumbe, Sezela \\
50 & 30.93 & -30.01 & To just south of Isipingo Beach & Scottburgh, Park Rynie, Umkomaas, Illovo, Amanzimtoti \\
51 & 31.15 & -29.62 & To Desainagar & Durban, Umhlanga Rocks, Umdloti Beach \\
52 & 31.46 & -29.26 & To just north of Zinkwazi Beach & Westbrook, Ballito, Blythdale Beach \\
53 & 31.82 & -28.94 & To just east of Mtunzini & Tugela River, Dunn’s Reserve \\
54 & 32.21 & -28.70 & To Mbonambi Beach & Richards Bay \\
55 & 32.46 & -28.32 & To just north of First Rocks & Dawson’s Rocks, Cape St Lucia, St Lucia \\
56 & 32.59 & -27.87 & To Bhukwini & Mission Rocks, Cape Vidal, Leven Point \\
57 & 32.72 & -27.42 & To just north of Gobey’s Point & Liefeldts Rocks, Sodwana Bay \\
58 & 32.87 & -26.97 & To Kosimeer & Hulley Point, Black Rock \\

\end{longtable}
\end{center}
\end{tiny}

\newpage

\section*{\large{Appendix B}}

\subsection*{Spatial analysis background and code}

The intention of this section is to show the approach and \textbf{R}
scripts used to pull apart the spatial scales at which seaweed
assemblages are structured around the coast of South Africa.
Specifically, I wish to determine if these scales match those expressed
by the coastal thermal provinces and the ocean regime underpinned by the
Agulhas and Benguela Currents.

\subsubsection*{The data}

I use two data sets. The first, \(Y\), comprises distribution records of
846 macroalgal species within each of 58 × 50 km-long sections (Appendix
A) of the South African coast (updated from Bolton and Stegenga, 2002).
This represents \emph{ca}. 90\% of the known seaweed flora of South
Africa, but excludes some very small and/or very rare species for which
data are insufficient. The data are from verifiable literature sources
and John Bolton and Rob Anderson's own collections, assembled from
information collected by teams of phycologists over three decades
(Bolton, 1986; Bolton and Stegenga, 2002; De Clerck et al., 2005;
Stegenga et al., 1997). The second, \(E\), is a dataset of \emph{in
situ} coastal seawater temperatures (Smit et al., 2013) derived from
daily measurements over up to 40 years.

A third data set of explanatory variables --- the spatial variables
(\(S\)) --- is constructed as per the instructions in section
\emph{Preparation of spatial variables}, later on.

\subsubsection*{Setting up the analysis environment}

This is \textbf{R}, so first I need to find, install and load various
packages. Some of the packages will be available on CRAN and can be
accessed and installed in the usual way, but others will have to be
downloaded from \href{https://r-forge.r-project.org/R/?group_id=195}{R
Forge}.

\begin{Shaded}
\begin{Highlighting}[]
\KeywordTok{library}\NormalTok{(betapart)}
\KeywordTok{library}\NormalTok{(vegan)}
\KeywordTok{library}\NormalTok{(gridExtra)}
\KeywordTok{library}\NormalTok{(packfor)}
\KeywordTok{library}\NormalTok{(grid)}
\KeywordTok{library}\NormalTok{(gridBase)}
\KeywordTok{library}\NormalTok{(tidyr)}
\KeywordTok{library}\NormalTok{(spdep) }\CommentTok{# for dnearneigh() in PCNM.R}
\KeywordTok{library}\NormalTok{(AEM) }\CommentTok{# for moran.I.multi() in PCNM.R}
\CommentTok{# install.packages("devtools")}
\CommentTok{# install.packages("AEM", repos = "http://R-Forge.R-project.org")}
\CommentTok{# install.packages("packfor", repos = "http://R-Forge.R-project.org")}
\KeywordTok{source}\NormalTok{(}\StringTok{"functions/pcoa_all.R"}\NormalTok{)}
\KeywordTok{source}\NormalTok{(}\StringTok{"functions/PCNM.R"}\NormalTok{)}
\KeywordTok{source}\NormalTok{(}\StringTok{"functions/spatial_MEM.R"}\NormalTok{)}
\end{Highlighting}
\end{Shaded}

Now I get to the data. The first step involves the species table
(\(Y\)). First I compute the Sørensen dissimilarity and then I decompose
the dissimilarity into the `turnover' (β\textsubscript{sim}) and
`nestedness-resultant' (β\textsubscript{sne}) components (Baselga, 2010;
Baselga et al., 2013) using the \texttt{betapart.core()} and
\texttt{betapart.pair()} functions of the \textbf{betapart} package
(Baselga et al., 2013). These are placed into the matrices \(Y1\) and
\(Y2\). Optionally, I can apply a prinipal components analysis (PCA) on
\(Y\) to find the major patterns in the community data. In
\textbf{vegan} this is done using the \texttt{rda()} function and not
supplying the constraints (\emph{i.e.} the environment table, \(E\), or
the spatial table, \(S\)). The formal analysis will use the species data
in distance-based redundancy analyses (db-RDA as per \textbf{vegan}'s
\texttt{capscale()} function) by coupling them with \(E\) and \(S\).

\begin{Shaded}
\begin{Highlighting}[]
\CommentTok{# Read in the species data (note: on GitHub only the distance}
\CommentTok{# matrices obtained via 'beta.part' and 'beta.pair' (below) }
\CommentTok{# will be provided -- they are read in as 'Y1.Rdata' and 'Y2.Rdata';}
\CommentTok{# the raw data cannot be shared at this stage, but the distance matrix is provided):}
\CommentTok{# spp <- read.csv('../stats/seaweeds.csv')}
\CommentTok{# spp <- dplyr::select(spp, -1)}

\CommentTok{# Decompose total Sørensen dissimilarity into turnover and }
\CommentTok{# nestedness-resultant components:}
\CommentTok{# Y.core <- betapart.core(spp) }
\CommentTok{# Y.pair <- beta.pair(Y.core, index.family = "sor")}

\CommentTok{# Let Y1 be the turnover component (beta-sim):}
\CommentTok{# Y1 <- as.matrix(Y.pair$beta.sim)}
\CommentTok{# save(Y1, file = "data/Y1.Rdata")}
\KeywordTok{load}\NormalTok{(}\StringTok{"data/Y1.Rdata"}\NormalTok{)}

\CommentTok{# Let Y2 be the nestedness-resultant component (beta-sne):}
\CommentTok{# Y2 <- as.matrix(Y.pair$beta.sne)}
\CommentTok{# save(Y2, file = "data/Y2.Rdata")}
\KeywordTok{load}\NormalTok{(}\StringTok{"data/Y2.Rdata"}\NormalTok{)}
\end{Highlighting}
\end{Shaded}

It is now necessary to load the environmental data and some setup files
that partition the 58 coastal sections (and the species and
environmental data that fall within these sections) into bioregions.

The thermal (environmental) data contain various variables, but in the
analysis I use only some of them. These data were obtained from many
sites along the South African coast, but using interpolation (not
included here) I calculated the thermal properties for each of the
coastal sections for which seaweed data are available. Consequently we
have a data frame with 58 rows and a column for each of the thermal
metrics. Before use, I apply \textbf{vegan}'s \texttt{decostand()}
function to scale the data to zero mean and unit variance.

Four bioregions are recognised for South Africa (Bolton and Anderson,
2004), namely the Benguela Marine Province (BMP; coastal sections
\textbf{1}--\textbf{17}), the Benguela-Agulhas Transition Zone (B-ATZ;
\textbf{18}--\textbf{22}), the Agulhas Marine Province (AMP;
\textbf{19}--\textbf{43}/\textbf{44}) and the East Coast Transition Zone
(ECTZ; \textbf{44}/\textbf{45}--\textbf{58}). My plotting functions
partition the data into the bioregions and colour code the figures
accordingly so I can see regional patterns in \textbeta-diversity
emerging.

\begin{Shaded}
\begin{Highlighting}[]
\CommentTok{# Now comes in the in situ temperatures for the 58 coastal sections (interpolated temperaures as per version 2 of the South African Coastal Temperature Network):}
\KeywordTok{load}\NormalTok{(}\StringTok{'data/E.RData'}\NormalTok{)}
\NormalTok{env <-}\StringTok{ }\KeywordTok{as.data.frame}\NormalTok{(interpOut)}

\CommentTok{# I select only some of the thermal vars; the rest}
\CommentTok{# are collinear with some of the ones I import:}
\NormalTok{E1 <-}\StringTok{ }\NormalTok{dplyr}\OperatorTok{::}\KeywordTok{select}\NormalTok{(env, febMean, febRange, febSD, augMean,}
\NormalTok{                    augRange, augSD, annMean, annRange, annSD)}

\CommentTok{# Calculate z-scores:}
\NormalTok{E1 <-}\StringTok{ }\KeywordTok{decostand}\NormalTok{(E1, }\DataTypeTok{method =} \StringTok{"standardize"}\NormalTok{)}

\CommentTok{# Load the coordinates of the coastal sections:}
\NormalTok{sites <-}\StringTok{ }\KeywordTok{read.csv}\NormalTok{(}\StringTok{"data/sites.csv"}\NormalTok{)}
\NormalTok{sites <-}\StringTok{ }\NormalTok{sites[, }\KeywordTok{c}\NormalTok{(}\DecValTok{2}\NormalTok{, }\DecValTok{1}\NormalTok{)]}

\CommentTok{# Load the bioregion definition:}
\NormalTok{bioreg <-}\StringTok{ }\KeywordTok{read.csv}\NormalTok{(}\StringTok{'data/bioregions.csv'}\NormalTok{, }\DataTypeTok{header =} \OtherTok{TRUE}\NormalTok{)}
\end{Highlighting}
\end{Shaded}

\subsubsection*{Preparation of spatial variables}

I test the niche difference mechanism as the primary species
compositional assembly process operating along South African shores. I
suggest that the thermal gradient along the coast provides a suite of
abiotic (thermal) conditions from which species can select based on
their physiological tolerances, and hence this will structure
\textbeta-diversity. For this mechanism to function one would assume
that all species have equal access to all sections along this stretch of
coast, thus following Beijerinck's `Law' that everything is everywhere
but the environment selects (Sauer, 1988) (but see main text!).

The basic approach to a spatial analysis structured around a biological
response (\emph{e.g.} community structure and composition; \(Y\)),
environmental variables (\(E\)) and their spatial representation (\(S\))
involves an analysis of Moran's eigenvector maps (MEM), followed by
db-RDA and variance partitioning. Various literature sources discuss
principle behind Moran's eigenvector maps (Dray et al., 2006, 2012).
Worked examples are also presented in the excellent book \emph{Numerical
Ecology with R} (Borcard et al., 2011) in Section 7.4. The method
followed here has been adapted from these and other sources.

Obtaining the MEMs to use in the analysis is based on the procedure
introduced by Borcard and Legendre (2002), which was later modified by
Dray et al. (2006). The basic approach involves:

\begin{enumerate}
\def\labelenumi{\arabic{enumi}.}
\item
  Set up a geographic or Euclidian distance matrix representing the
  pairwise distances between the \(n\) sites (\(D=[d_{ij}]\)). I already
  did this when I applied the \texttt{decostand} function earlier.
\item
  Construct a truncated distance matrix by calculating a Minimum
  Spanning Tree (\(S^{\star}\)) and noting the following rules:
  \[S^{\star} =\left\{ \begin{array}{rl} 0 & \mbox{if}~i = j \\ d_{ij} & \mbox{if}~d_{ij} \leq t \\ 4t & \mbox{if}~d_{ij} > t \end{array} \right.\]
  Weighting may be applied if desired, resulting in \(S^{\star}_w\). It
  is not done here.
\item
  Do a Principal Coordinates Analysis (PCoA) of the truncated distance
  matrix \(S^{\star}\).
\end{enumerate}

The spatial properties imprinted on the species and their environment
can be specified using a matrix of Euclidian or geographic distances.
These coordinates are `truncated' into a square (section \(\times\)
section) matrix containing non-negative values (\(S^{\star}\)). By
convention the diagonal values are set to zero. A very basic spatial
matrix is binary, where 1 codes for pairs of neigbouring sites while 0
denotes non-connected sites according to the chosen network topology.
Such matrices are called `binary connectivity matrices' and relate to
graphs made using distance criteria derived from graph theory.

Truncation produced by Minimum Spanning Trees (MST) focuses on the
binary relationships between neighbouring sites, discarding any other
connections (\emph{i.e.} some sites are considered to be neighbours,
while for others the relationships are null). One could also choose a
Gabriel graph or another kind of network topology. Such matrix
representations show section-to-section connectivities. In the case of
South Africa's coastline data, the MST causes sections to be connected
only to other sections adjacent to two sides of it: for example, Section
\textbf{4} is directly connected to \emph{only} Sections \textbf{3} and
\textbf{5}; sections at the termini of the coastal `string' of sections
are each connected to only one other section. The binary connectivity
matrices, also called \emph{topology-based connectivity matrices}, can
be produced from Euclidian or geographic coordinates using functions in
at least two \textbf{R} packages (I start with geographic coordinates).
One option is to use the \textbf{spdep} package's \texttt{mst.nb()}
function to calculate a MST, but there are also options in the
\textbf{vegan} package and elsewhere. The neighbours list arrived at
from the MST represents the spatial component, \(S^{\star}\). The MST
results in small connectivity artefacts in the Saldanha Bay region where
the closest sections are not necessarily the ones adjacent one-another
following along the path around the coast, because sections at opposite
sides of the bay may in fact be closer together. This topological
inconsistency does not affect the spatial analysis in any way.

Once the truncated distance matrix has been prepared, it is subjected to
a PCoA and I keep the eigenvectors that represent positive spatial
correlation (positive Moran's \(I\)). For the MEM analysis I use the
function \texttt{PCNM()} that resides in the \texttt{functions} folder
in the file \texttt{PCNM.R} (see notes inside about authorship). PCNM
stands for Principal Coordinates Analysis of Neighbourhood Matrices (the
neighbourhood matrix in this instance being the MST). This method
automatically constructs the spatial variables and calculates the
Moran's I for each. The MEMs are completely orthogonal and represent the
spatial structures over the full range of scales from 50 to 2,700 km.
The large eigenvectors represent broad spatial scales while smaller ones
cover finer features. The \emph{spatial data} will be used as a set of
explanatory variables in the multiple regression type analyses applied
to a species dissimilarity matrix (\emph{i.e.} the db-RDA; Dray et al.,
2012)

The code below reproduces the spatial analysis in the paper. Due to the
length of the output I have prevented the script from returning any
output here; rather, if the reader is for some odd reason interested in
repeating this analysis, s/he may find the data and scripts in my
\href{https://github.com/ajsmit/Seaweed-beta}{GitHub} repository, and
the full code can be run in its entirety. Well, I hope this will work,
but if it doesn't (probably very likely) then write to me at
\href{mailto:ajsmit@uwc.ac.za}{\nolinkurl{ajsmit@uwc.ac.za}} and I shall
assist --- this may depend on if your email has a catchy title that will
make it stand out from among all the other emails which other people
think are equally important.

\begin{Shaded}
\begin{Highlighting}[]
\NormalTok{## Auto PCNM:}
\NormalTok{S.auto <-}\StringTok{ }\KeywordTok{PCNM}\NormalTok{(}\KeywordTok{dist}\NormalTok{(sites), }\DataTypeTok{silent =} \OtherTok{TRUE}\NormalTok{)}
\CommentTok{# summary(S.auto)}

\CommentTok{# The truncation distance:}
\NormalTok{S.dmin <-}\StringTok{ }\NormalTok{S.auto}\OperatorTok{$}\NormalTok{thresh}

\CommentTok{# The number of eigenvalues:}
\NormalTok{S.len <-}\StringTok{ }\KeywordTok{length}\NormalTok{(S.auto}\OperatorTok{$}\NormalTok{values)}

\CommentTok{# Expected value of I, no spatial correlation:}
\NormalTok{S.auto}\OperatorTok{$}\NormalTok{expected_Moran}

\CommentTok{# Select eigenfunction with positive spatial correlation:}
\NormalTok{S.sel <-}\StringTok{ }\KeywordTok{which}\NormalTok{(S.auto}\OperatorTok{$}\NormalTok{Moran_I}\OperatorTok{$}\NormalTok{Positive }\OperatorTok{==}\StringTok{ }\OtherTok{TRUE}\NormalTok{)}
\CommentTok{# length(S.sel)}
\CommentTok{# there are 27 MEMs, i.e. 27 of the PCNM variables (eigenvalues) relate}
\CommentTok{# significantly to Moran's I}

\CommentTok{# Extract the eigenvectors associated with those MEMs:}
\NormalTok{S.pos <-}\StringTok{ }\KeywordTok{as.data.frame}\NormalTok{(S.auto}\OperatorTok{$}\NormalTok{vectors)[, S.sel]}
\end{Highlighting}
\end{Shaded}

The code below lets us visualise the configuration of the 58 coastal
sections as represented by the minimum spanning tree. Because the sites
are constrained by the coast the MST network topology results in a
string of coastal sections arranged along the shore between Section
\textbf{1} and Section \textbf{58}. This spatial network therefore also
captures the spatial connectivity in the seaweed's dispersal ability
along the shore, although no directionality is associated with
dispersal. In the paper I discuss the possible influence of ocean
currents (\emph{e.g.} Wernberg et al., 2013) and I pointed out that it
is tempting to assume that seaweeds would disperse in the direction the
major ocean currents. These kinds of networks could conceivably be
configured to model dispersal due to currents, but here it is simply
used for representing the spatial scale of the study region.

\begin{Shaded}
\begin{Highlighting}[]
\CommentTok{# The spatial netwwork topology of the coastal sections can be seen by:}
\KeywordTok{plot}\NormalTok{(S.auto}\OperatorTok{$}\NormalTok{spanning, sites)}
\end{Highlighting}
\end{Shaded}

\subsubsection*{db-RDA on the MEMs}

The next step of the spatial analysis is to apply a db-RDA with the
seaweed data (\(Y1\) and \(Y2\)) coupled with the MEMs. I now run a full
(global) db-RDA on the significant, positive MEMs selected above, and I
then perform a permutation test to see if the fit is significant.

\begin{Shaded}
\begin{Highlighting}[]
\CommentTok{# Run the db-RDA on the Y1 data:}
\NormalTok{S.Y1.cs <-}\StringTok{ }\KeywordTok{capscale}\NormalTok{(Y1 }\OperatorTok{~}\NormalTok{., S.pos)}

\CommentTok{# Permutation test to test for the significance of the global fit:}
\KeywordTok{anova}\NormalTok{(S.Y1.cs, }\DataTypeTok{parallel =} \DecValTok{4}\NormalTok{) }\CommentTok{# ... yes, significant!}

\CommentTok{# The global adjusted R2 --- the variance explained by the constrained axes:}
\NormalTok{S.Y1.cs.R2 <-}\StringTok{ }\KeywordTok{RsquareAdj}\NormalTok{(S.Y1.cs)}\OperatorTok{$}\NormalTok{adj.r.squared}

\CommentTok{# Variance explained by full model:}
\KeywordTok{sum}\NormalTok{(S.Y1.cs}\OperatorTok{$}\NormalTok{CCA}\OperatorTok{$}\NormalTok{eig) }\OperatorTok{/}\StringTok{ }\NormalTok{S.Y1.cs}\OperatorTok{$}\NormalTok{tot.chi }\OperatorTok{*}\StringTok{ }\DecValTok{100}
\end{Highlighting}
\end{Shaded}

\begin{Shaded}
\begin{Highlighting}[]
\CommentTok{# And on the Y2 data (uncommented, but same as above):}
\NormalTok{S.Y2.cs <-}\StringTok{ }\KeywordTok{capscale}\NormalTok{(Y2 }\OperatorTok{~}\NormalTok{., S.pos)}
\NormalTok{S.Y2.cs.R2 <-}\StringTok{ }\KeywordTok{RsquareAdj}\NormalTok{(S.Y2.cs)}\OperatorTok{$}\NormalTok{adj.r.squared}
\KeywordTok{sum}\NormalTok{(S.Y2.cs}\OperatorTok{$}\NormalTok{CCA}\OperatorTok{$}\NormalTok{eig) }\OperatorTok{/}\StringTok{ }\NormalTok{S.Y2.cs}\OperatorTok{$}\NormalTok{tot.chi }\OperatorTok{*}\StringTok{ }\DecValTok{100}
\end{Highlighting}
\end{Shaded}

Since the analysis is significant, I compute the adjusted
\emph{R}\textsuperscript{2} and run forward selection of the MEMs. The
forward selection procedure of Blanchet et al. (2008) is implemented in
the \textbf{packfor} package for R, and I use it to reduce the number of
MEM variables and retain only those that best fit the biotic data.
Forward selection prevents the inflation of the overall type I error and
reduces the number of explanatory variables used in the final model,
which improves parsimony. I then run a new db-RDA analysis on the `best'
(reduced) set of MEM variables that was selected.

\begin{Shaded}
\begin{Highlighting}[]
\CommentTok{# Forward selection on Y1:}
\NormalTok{S.Y1.fwd <-}\StringTok{ }\KeywordTok{forward.sel}\NormalTok{(Y1, }\KeywordTok{as.matrix}\NormalTok{(S.pos), }\DataTypeTok{adjR2thresh =}\NormalTok{ S.Y1.cs.R2)}

\CommentTok{# Forward selection on Y2:}
\NormalTok{S.Y2.fwd <-}\StringTok{ }\KeywordTok{forward.sel}\NormalTok{(Y2, }\KeywordTok{as.matrix}\NormalTok{(S.pos), }\DataTypeTok{adjR2thresh =}\NormalTok{ S.Y2.cs.R2)}
\end{Highlighting}
\end{Shaded}

\begin{Shaded}
\begin{Highlighting}[]
\CommentTok{# Write the significant MEMs to a new object:}
\NormalTok{S.Y1.no.sig <-}\StringTok{ }\KeywordTok{nrow}\NormalTok{(S.Y1.fwd)}
\NormalTok{S.Y1.sign <-}\StringTok{ }\KeywordTok{sort}\NormalTok{(S.Y1.fwd[, }\DecValTok{2}\NormalTok{])}
\NormalTok{S.Y1.red <-}\StringTok{ }\NormalTok{S.pos[, }\KeywordTok{c}\NormalTok{(S.Y1.sign)]}
\KeywordTok{colnames}\NormalTok{(S.Y1.red) <-}\StringTok{ }\KeywordTok{paste}\NormalTok{(}\KeywordTok{rep}\NormalTok{(}\StringTok{"MEM"}\NormalTok{, S.Y1.no.sig),}
                            \KeywordTok{as.character}\NormalTok{(S.Y1.sign), }\DataTypeTok{sep =} \StringTok{""}\NormalTok{)}

\CommentTok{# Identity of significant MEMs:}
\KeywordTok{colnames}\NormalTok{(S.Y1.red)}

\CommentTok{# Run a new db-RDA on the best MEM variables:}
\NormalTok{S.Y1.s2 <-}\StringTok{ }\KeywordTok{capscale}\NormalTok{(Y1 }\OperatorTok{~}\NormalTok{., }\DataTypeTok{data =}\NormalTok{ S.Y1.red)}
\CommentTok{# no need to check these for collinearity as the }
\CommentTok{# MEMs are completely orthogonal..}

\CommentTok{# Permutation test to test for significance:}
\KeywordTok{anova}\NormalTok{(S.Y1.s2, }\DataTypeTok{parallel =} \DecValTok{4}\NormalTok{)}

\CommentTok{# Test by axis:}
\KeywordTok{anova}\NormalTok{(S.Y1.s2, }\DataTypeTok{by =} \StringTok{"axis"}\NormalTok{, }\DataTypeTok{parallel =} \DecValTok{4}\NormalTok{)}

\CommentTok{# The significant axes:}
\NormalTok{S.Y1.axis.test <-}\StringTok{ }\KeywordTok{anova}\NormalTok{(S.Y1.s2, }\DataTypeTok{by =} \StringTok{"terms"}\NormalTok{, }\DataTypeTok{parallel =} \DecValTok{4}\NormalTok{)}
\NormalTok{S.Y1.ax <-}\StringTok{ }\KeywordTok{which}\NormalTok{(S.Y1.axis.test[, }\DecValTok{4}\NormalTok{] }\OperatorTok{<}\StringTok{ }\FloatTok{0.05}\NormalTok{)}
\NormalTok{S.Y1.sign.ax <-}\StringTok{ }\KeywordTok{colnames}\NormalTok{(S.Y1.red[,S.Y1.ax])}

\CommentTok{# Test by terms:}
\KeywordTok{anova}\NormalTok{(S.Y1.s2, }\DataTypeTok{by =} \StringTok{"terms"}\NormalTok{, }\DataTypeTok{parallel =} \DecValTok{4}\NormalTok{)}

\CommentTok{# The adjusted R2 --- the variance explained by the constrained axes:}
\NormalTok{S.Y1.s2.R2 <-}\StringTok{ }\KeywordTok{RsquareAdj}\NormalTok{(S.Y1.s2)}\OperatorTok{$}\NormalTok{adj.r.squared}

\CommentTok{# Variance explained by reduced model:}
\KeywordTok{sum}\NormalTok{(S.Y1.s2}\OperatorTok{$}\NormalTok{CCA}\OperatorTok{$}\NormalTok{eig) }\OperatorTok{/}\StringTok{ }\NormalTok{S.Y1.s2}\OperatorTok{$}\NormalTok{tot.chi }\OperatorTok{*}\StringTok{ }\DecValTok{100}

\CommentTok{# Show only the first 6 rows:}
\KeywordTok{scores}\NormalTok{(S.Y1.s2, }\DataTypeTok{display =} \StringTok{"bp"}\NormalTok{, }\DataTypeTok{choices =} \KeywordTok{c}\NormalTok{(}\DecValTok{1}\OperatorTok{:}\DecValTok{4}\NormalTok{))[}\DecValTok{1}\OperatorTok{:}\DecValTok{6}\NormalTok{, ]}
\end{Highlighting}
\end{Shaded}

\begin{Shaded}
\begin{Highlighting}[]
\CommentTok{# As above, but now with Y2:}
\NormalTok{S.Y2.no.sig <-}\StringTok{ }\KeywordTok{nrow}\NormalTok{(S.Y2.fwd)}
\NormalTok{S.Y2.sign <-}\StringTok{ }\KeywordTok{sort}\NormalTok{(S.Y2.fwd[, }\DecValTok{2}\NormalTok{])}
\NormalTok{S.Y2.red <-}\StringTok{ }\NormalTok{S.pos[, }\KeywordTok{c}\NormalTok{(S.Y2.sign)]}
\KeywordTok{colnames}\NormalTok{(S.Y2.red) <-}\StringTok{ }\KeywordTok{paste}\NormalTok{(}\KeywordTok{rep}\NormalTok{(}\StringTok{"MEM"}\NormalTok{, S.Y2.no.sig),}
                            \KeywordTok{as.character}\NormalTok{(S.Y2.sign), }\DataTypeTok{sep =} \StringTok{""}\NormalTok{)}
\KeywordTok{colnames}\NormalTok{(S.Y2.red)}
\NormalTok{S.Y2.s2 <-}\StringTok{ }\KeywordTok{capscale}\NormalTok{(Y2 }\OperatorTok{~}\NormalTok{., }\DataTypeTok{data =}\NormalTok{ S.Y2.red)}

\KeywordTok{anova}\NormalTok{(S.Y2.s2, }\DataTypeTok{parallel =} \DecValTok{4}\NormalTok{) }\CommentTok{# ... yes, significant!}

\KeywordTok{anova}\NormalTok{(S.Y2.s2, }\DataTypeTok{by =} \StringTok{"axis"}\NormalTok{, }\DataTypeTok{parallel =} \DecValTok{4}\NormalTok{)}

\NormalTok{S.Y2.axis.test <-}\StringTok{ }\KeywordTok{anova}\NormalTok{(S.Y2.s2, }\DataTypeTok{by =} \StringTok{"terms"}\NormalTok{, }\DataTypeTok{parallel =} \DecValTok{4}\NormalTok{)}
\NormalTok{S.Y2.ax <-}\StringTok{ }\KeywordTok{which}\NormalTok{(S.Y2.axis.test[, }\DecValTok{4}\NormalTok{] }\OperatorTok{<}\StringTok{ }\FloatTok{0.05}\NormalTok{)}
\NormalTok{S.Y2.sign.ax <-}\StringTok{ }\KeywordTok{colnames}\NormalTok{(S.Y2.red[,S.Y2.ax])}

\NormalTok{S.Y2.s2.R2 <-}\StringTok{ }\KeywordTok{RsquareAdj}\NormalTok{(S.Y2.s2)}\OperatorTok{$}\NormalTok{adj.r.squared}

\KeywordTok{sum}\NormalTok{(S.Y2.s2}\OperatorTok{$}\NormalTok{CCA}\OperatorTok{$}\NormalTok{eig) }\OperatorTok{/}\StringTok{ }\NormalTok{S.Y2.s2}\OperatorTok{$}\NormalTok{tot.chi }\OperatorTok{*}\StringTok{ }\DecValTok{100}

\KeywordTok{scores}\NormalTok{(S.Y2.s2, }\DataTypeTok{display =} \StringTok{"bp"}\NormalTok{, }\DataTypeTok{choices =} \KeywordTok{c}\NormalTok{(}\DecValTok{1}\OperatorTok{:}\DecValTok{4}\NormalTok{))}
\end{Highlighting}
\end{Shaded}

\subsubsection*{A few visualisations}

Now I make a visualisation to reveal the spatial arrangement of the MEMs
used in the final db-RDA involving the spatial variables (\emph{i.e.}
\texttt{S.Y1.s2} and \texttt{S.Y2.s2}). The spatial configuration
relates to broad scales as seen in Fig. 3 in the paper. Here are plots
of the site scores for the MEMs and \(Y1\) and \(Y2\) (a few panels
belonging with Fig. 3):

\begin{Shaded}
\begin{Highlighting}[]
\CommentTok{# Plot the first canonical axis of the db-RDA with the significant MEMs for Y1;}
\CommentTok{# (see Fig. 3):}
\NormalTok{S.Y1.axes <-}\StringTok{ }\KeywordTok{scores}\NormalTok{(S.Y1.s2, }\DataTypeTok{choices =} \KeywordTok{c}\NormalTok{(}\DecValTok{1}\OperatorTok{:}\DecValTok{3}\NormalTok{), }\DataTypeTok{display =} \StringTok{"lc"}\NormalTok{, }\DataTypeTok{scaling =} \DecValTok{1}\NormalTok{)}
\NormalTok{S.Y1.plt.axis1 <-}\StringTok{ }\KeywordTok{ggmap}\NormalTok{() }\OperatorTok{+}
\StringTok{  }\KeywordTok{geom_point}\NormalTok{(}\DataTypeTok{data =}\NormalTok{ sites, }\KeywordTok{aes}\NormalTok{(}\DataTypeTok{x =}\NormalTok{ Longitude, }\DataTypeTok{y =}\NormalTok{ Latitude,}
                               \DataTypeTok{size =} \KeywordTok{abs}\NormalTok{(S.Y1.axes[, }\DecValTok{1}\NormalTok{]),}
                               \DataTypeTok{col =} \KeywordTok{ifelse}\NormalTok{(S.Y1.axes[, }\DecValTok{1}\NormalTok{] }\OperatorTok{<}\StringTok{ }\DecValTok{0}\NormalTok{, }\StringTok{"a"}\NormalTok{, }\StringTok{"b"}\NormalTok{)), }\DataTypeTok{shape =} \DecValTok{1}\NormalTok{) }\OperatorTok{+}
\StringTok{  }\KeywordTok{scale_size_continuous}\NormalTok{(}\DataTypeTok{guide =} \OtherTok{FALSE}\NormalTok{) }\OperatorTok{+}
\StringTok{  }\KeywordTok{scale_colour_manual}\NormalTok{(}\DataTypeTok{guide =} \OtherTok{FALSE}\NormalTok{, }\DataTypeTok{values =} \KeywordTok{c}\NormalTok{(}\StringTok{"black"}\NormalTok{, }\StringTok{"grey60"}\NormalTok{)) }\OperatorTok{+}
\StringTok{  }\KeywordTok{ggtitle}\NormalTok{(}\KeywordTok{expression}\NormalTok{(}\KeywordTok{paste}\NormalTok{(}\StringTok{"CAP1 of spatial variables, "}\NormalTok{, beta[sim])))}

\CommentTok{# And the same for Y2 (see Fig. 3):}
\NormalTok{S.Y2.axes <-}\StringTok{ }\KeywordTok{scores}\NormalTok{(S.Y2.s2, }\DataTypeTok{choices =} \KeywordTok{c}\NormalTok{(}\DecValTok{1}\OperatorTok{:}\DecValTok{3}\NormalTok{), }\DataTypeTok{display =} \StringTok{"lc"}\NormalTok{, }\DataTypeTok{scaling =} \DecValTok{1}\NormalTok{)}
\NormalTok{S.Y2.plt.axis1 <-}\StringTok{ }\KeywordTok{ggmap}\NormalTok{() }\OperatorTok{+}
\StringTok{  }\KeywordTok{geom_point}\NormalTok{(}\DataTypeTok{data =}\NormalTok{ sites, }\KeywordTok{aes}\NormalTok{(}\DataTypeTok{x =}\NormalTok{ Longitude, }\DataTypeTok{y =}\NormalTok{ Latitude,}
                               \DataTypeTok{size =} \KeywordTok{abs}\NormalTok{(S.Y2.axes[, }\DecValTok{1}\NormalTok{]),}
                               \DataTypeTok{col =} \KeywordTok{ifelse}\NormalTok{(S.Y2.axes[, }\DecValTok{1}\NormalTok{] }\OperatorTok{<}\StringTok{ }\DecValTok{0}\NormalTok{, }\StringTok{"a"}\NormalTok{, }\StringTok{"b"}\NormalTok{)), }\DataTypeTok{shape =} \DecValTok{1}\NormalTok{) }\OperatorTok{+}
\StringTok{  }\KeywordTok{scale_size_continuous}\NormalTok{(}\DataTypeTok{guide =} \OtherTok{FALSE}\NormalTok{) }\OperatorTok{+}
\StringTok{  }\KeywordTok{scale_colour_manual}\NormalTok{(}\DataTypeTok{guide =} \OtherTok{FALSE}\NormalTok{, }\DataTypeTok{values =} \KeywordTok{c}\NormalTok{(}\StringTok{"black"}\NormalTok{, }\StringTok{"grey60"}\NormalTok{)) }\OperatorTok{+}
\StringTok{  }\KeywordTok{ggtitle}\NormalTok{(}\KeywordTok{expression}\NormalTok{(}\KeywordTok{paste}\NormalTok{(}\StringTok{"CAP1 of spatial variables, "}\NormalTok{, beta[sne])))}
\end{Highlighting}
\end{Shaded}

Now that I know that spatial structures are present in the seaweed data
I check how these significant spatial patterns (two significant
canonical axes, CAP1 and CAP2) are related to the environmental
variables using linear regression. Checks for normality are also done
but none of the output is printed here.

Next I want to show the ordination biplots of the MEM variables with
respect to the sites using scaling = 2 (species) and showing the LC
scores. Now I can see the major directions of influence of the spatial
variables with respect to the sites. The code below produces a few
panels of Fig. 2:

\begin{Shaded}
\begin{Highlighting}[]
\CommentTok{# A few of the panels that go with Fig. 2;}
\CommentTok{# first for Y1...:}
\NormalTok{S.Y1.scrs <-}\StringTok{ }\KeywordTok{scores}\NormalTok{(S.Y1.s2, }\DataTypeTok{display =} \KeywordTok{c}\NormalTok{(}\StringTok{"sp"}\NormalTok{,}\StringTok{"wa"}\NormalTok{,}\StringTok{"lc"}\NormalTok{,}\StringTok{"bp"}\NormalTok{,}\StringTok{"cn"}\NormalTok{))}
\NormalTok{S.Y1.df_sites <-}\StringTok{ }\KeywordTok{data.frame}\NormalTok{(S.Y1.scrs}\OperatorTok{$}\NormalTok{constraints)}
\NormalTok{S.Y1.df_sites}\OperatorTok{$}\NormalTok{bioreg <-}\StringTok{ }\NormalTok{bioreg}\OperatorTok{$}\NormalTok{bolton}
\NormalTok{S.Y1.df_sites}\OperatorTok{$}\NormalTok{section <-}\StringTok{ }\KeywordTok{seq}\NormalTok{(}\DecValTok{1}\OperatorTok{:}\DecValTok{58}\NormalTok{)}
\KeywordTok{colnames}\NormalTok{(S.Y1.df_sites) <-}\StringTok{ }\KeywordTok{c}\NormalTok{(}\StringTok{"x"}\NormalTok{, }\StringTok{"y"}\NormalTok{, }\StringTok{"Bioregion"}\NormalTok{, }\StringTok{"Section"}\NormalTok{)}

\NormalTok{multiplier <-}\StringTok{ }\KeywordTok{ordiArrowMul}\NormalTok{(S.Y1.scrs}\OperatorTok{$}\NormalTok{biplot)}
\NormalTok{S.Y1.bp <-}\StringTok{ }\NormalTok{S.Y1.scrs}\OperatorTok{$}\NormalTok{biplot }\OperatorTok{*}\StringTok{ }\NormalTok{multiplier}
\NormalTok{S.Y1.bp <-}\StringTok{ }\KeywordTok{as.data.frame}\NormalTok{(S.Y1.bp)}
\NormalTok{S.Y1.bp}\OperatorTok{$}\NormalTok{labels <-}\StringTok{ }\KeywordTok{rownames}\NormalTok{(S.Y1.bp)}
\KeywordTok{colnames}\NormalTok{(S.Y1.bp) <-}\StringTok{ }\KeywordTok{c}\NormalTok{(}\StringTok{"x"}\NormalTok{, }\StringTok{"y"}\NormalTok{, }\StringTok{"labels"}\NormalTok{)}
\NormalTok{S.Y1.bp.sign <-}\StringTok{ }\NormalTok{S.Y1.bp[S.Y1.bp}\OperatorTok{$}\NormalTok{labels }\OperatorTok\StringTok{ }\NormalTok{S.Y1.sign.ax,]}

\CommentTok{# A modification of the vegan ordiArrowTextXY() function to prevent the }
\CommentTok{# "plot.new has not been called yet" from occuring}
\KeywordTok{source}\NormalTok{(}\StringTok{"functions/text_mult.R"}\NormalTok{)}

\NormalTok{S.Y1.text <-}\StringTok{ }\KeywordTok{text.mult}\NormalTok{(S.Y1.scrs}\OperatorTok{$}\NormalTok{biplot)}
\NormalTok{S.Y1.text <-}\StringTok{ }\KeywordTok{as.data.frame}\NormalTok{(S.Y1.text)}
\NormalTok{S.Y1.text}\OperatorTok{$}\NormalTok{labels <-}\StringTok{ }\KeywordTok{rownames}\NormalTok{(S.Y1.text)}
\KeywordTok{colnames}\NormalTok{(S.Y1.text) <-}\StringTok{ }\KeywordTok{c}\NormalTok{(}\StringTok{"x"}\NormalTok{, }\StringTok{"y"}\NormalTok{, }\StringTok{"labels"}\NormalTok{)}
\NormalTok{S.Y1.text.sign <-}\StringTok{ }\NormalTok{S.Y1.text[S.Y1.text}\OperatorTok{$}\NormalTok{labels }\OperatorTok\StringTok{ }\NormalTok{S.Y1.sign.ax,]}

\NormalTok{S.Y1.p <-}\StringTok{ }\KeywordTok{ggplot}\NormalTok{(}\DataTypeTok{data =}\NormalTok{ S.Y1.df_sites, }\KeywordTok{aes}\NormalTok{(x, y, }\DataTypeTok{colour =}\NormalTok{ Bioregion)) }\OperatorTok{+}\StringTok{ }
\StringTok{  }\KeywordTok{geom_point}\NormalTok{(}\DataTypeTok{size =} \FloatTok{4.0}\NormalTok{) }\OperatorTok{+}\StringTok{ }
\StringTok{  }\KeywordTok{geom_text}\NormalTok{(}\KeywordTok{aes}\NormalTok{(}\DataTypeTok{label =}\NormalTok{ Section), }\DataTypeTok{size =} \FloatTok{3.0}\NormalTok{, }\DataTypeTok{col =} \StringTok{"white"}\NormalTok{) }\OperatorTok{+}\StringTok{ }
\StringTok{  }\KeywordTok{geom_segment}\NormalTok{(}\DataTypeTok{data =}\NormalTok{ S.Y1.bp, }
               \KeywordTok{aes}\NormalTok{(}\DataTypeTok{x =} \DecValTok{0}\NormalTok{, }\DataTypeTok{y =} \DecValTok{0}\NormalTok{, }\DataTypeTok{xend =}\NormalTok{ x, }\DataTypeTok{yend =}\NormalTok{ y),}
               \DataTypeTok{arrow =} \KeywordTok{arrow}\NormalTok{(}\DataTypeTok{length =} \KeywordTok{unit}\NormalTok{(}\FloatTok{0.2}\NormalTok{, }\StringTok{"cm"}\NormalTok{)), }
               \DataTypeTok{color =} \StringTok{"red"}\NormalTok{, }\DataTypeTok{alpha =} \DecValTok{1}\NormalTok{, }\DataTypeTok{size =} \FloatTok{0.7}\NormalTok{) }\OperatorTok{+}
\StringTok{  }\KeywordTok{geom_text}\NormalTok{(}\DataTypeTok{data =} \KeywordTok{as.data.frame}\NormalTok{(S.Y1.text), }
            \KeywordTok{aes}\NormalTok{(x, y, }\DataTypeTok{label =} \KeywordTok{rownames}\NormalTok{(S.Y1.text)),}
            \DataTypeTok{color =} \StringTok{"black"}\NormalTok{) }\OperatorTok{+}
\StringTok{  }\KeywordTok{xlab}\NormalTok{(}\StringTok{"CAP1"}\NormalTok{) }\OperatorTok{+}\StringTok{ }\KeywordTok{ylab}\NormalTok{(}\StringTok{"CAP2"}\NormalTok{) }\OperatorTok{+}\StringTok{ }
\StringTok{  }\KeywordTok{ggtitle}\NormalTok{(}\KeywordTok{expression}\NormalTok{(}\KeywordTok{paste}\NormalTok{(}\StringTok{"Spatial variables and "}\NormalTok{, beta[sim]))) }\OperatorTok{+}
\StringTok{  }\KeywordTok{theme_grey}\NormalTok{() }\OperatorTok{+}
\StringTok{  }\KeywordTok{theme}\NormalTok{(}\DataTypeTok{panel.grid.minor =} \KeywordTok{element_blank}\NormalTok{(),}
        \DataTypeTok{legend.position =} \StringTok{"none"}\NormalTok{,}
        \DataTypeTok{aspect.ratio =} \FloatTok{0.8}\NormalTok{)}

\CommentTok{# ...then for Y2:}
\NormalTok{S.Y2.scrs <-}\StringTok{ }\KeywordTok{scores}\NormalTok{(S.Y2.s2, }\DataTypeTok{display =} \KeywordTok{c}\NormalTok{(}\StringTok{"sp"}\NormalTok{,}\StringTok{"wa"}\NormalTok{,}\StringTok{"lc"}\NormalTok{,}\StringTok{"bp"}\NormalTok{,}\StringTok{"cn"}\NormalTok{))}
\NormalTok{S.Y2.df_sites <-}\StringTok{ }\KeywordTok{data.frame}\NormalTok{(S.Y2.scrs}\OperatorTok{$}\NormalTok{constraints)}
\NormalTok{S.Y2.df_sites}\OperatorTok{$}\NormalTok{bioreg <-}\StringTok{ }\NormalTok{bioreg}\OperatorTok{$}\NormalTok{bolton}
\NormalTok{S.Y2.df_sites}\OperatorTok{$}\NormalTok{section <-}\StringTok{ }\KeywordTok{seq}\NormalTok{(}\DecValTok{1}\OperatorTok{:}\DecValTok{58}\NormalTok{)}
\KeywordTok{colnames}\NormalTok{(S.Y2.df_sites) <-}\StringTok{ }\KeywordTok{c}\NormalTok{(}\StringTok{"x"}\NormalTok{, }\StringTok{"y"}\NormalTok{, }\StringTok{"Bioregion"}\NormalTok{, }\StringTok{"Section"}\NormalTok{)}

\NormalTok{multiplier <-}\StringTok{ }\KeywordTok{ordiArrowMul}\NormalTok{(S.Y2.scrs}\OperatorTok{$}\NormalTok{biplot, }\DataTypeTok{fill =} \FloatTok{0.25}\NormalTok{)}
\NormalTok{S.Y2.bp <-}\StringTok{ }\NormalTok{S.Y2.scrs}\OperatorTok{$}\NormalTok{biplot }\OperatorTok{*}\StringTok{ }\NormalTok{multiplier}
\NormalTok{S.Y2.bp <-}\StringTok{ }\KeywordTok{as.data.frame}\NormalTok{(S.Y2.bp)}
\NormalTok{S.Y2.bp}\OperatorTok{$}\NormalTok{labels <-}\StringTok{ }\KeywordTok{rownames}\NormalTok{(S.Y2.bp)}
\KeywordTok{colnames}\NormalTok{(S.Y2.bp) <-}\StringTok{ }\KeywordTok{c}\NormalTok{(}\StringTok{"x"}\NormalTok{, }\StringTok{"y"}\NormalTok{, }\StringTok{"labels"}\NormalTok{)}
\NormalTok{S.Y2.bp.sign <-}\StringTok{ }\NormalTok{S.Y2.bp[S.Y2.bp}\OperatorTok{$}\NormalTok{labels }\OperatorTok\StringTok{ }\NormalTok{S.Y2.sign.ax,]}

\NormalTok{S.Y2.text <-}\StringTok{ }\KeywordTok{text.mult}\NormalTok{(S.Y2.scrs}\OperatorTok{$}\NormalTok{biplot, }\DataTypeTok{fill =} \FloatTok{0.25}\NormalTok{)}
\NormalTok{S.Y2.text <-}\StringTok{ }\KeywordTok{as.data.frame}\NormalTok{(S.Y2.text)}
\NormalTok{S.Y2.text}\OperatorTok{$}\NormalTok{labels <-}\StringTok{ }\KeywordTok{rownames}\NormalTok{(S.Y2.text)}
\KeywordTok{colnames}\NormalTok{(S.Y2.text) <-}\StringTok{ }\KeywordTok{c}\NormalTok{(}\StringTok{"x"}\NormalTok{, }\StringTok{"y"}\NormalTok{, }\StringTok{"labels"}\NormalTok{)}
\NormalTok{S.Y2.text.sign <-}\StringTok{ }\NormalTok{S.Y2.text[S.Y2.text}\OperatorTok{$}\NormalTok{labels }\OperatorTok\StringTok{ }\NormalTok{S.Y2.sign.ax,]}

\NormalTok{S.Y2.p <-}\StringTok{ }\KeywordTok{ggplot}\NormalTok{(}\DataTypeTok{data =}\NormalTok{ S.Y2.df_sites, }\KeywordTok{aes}\NormalTok{(x, y, }\DataTypeTok{colour =}\NormalTok{ Bioregion)) }\OperatorTok{+}\StringTok{ }
\StringTok{  }\KeywordTok{geom_point}\NormalTok{(}\DataTypeTok{size =} \FloatTok{4.0}\NormalTok{) }\OperatorTok{+}\StringTok{ }
\StringTok{  }\KeywordTok{geom_text}\NormalTok{(}\KeywordTok{aes}\NormalTok{(}\DataTypeTok{label =}\NormalTok{ Section), }\DataTypeTok{size =} \FloatTok{3.0}\NormalTok{, }\DataTypeTok{col =} \StringTok{"white"}\NormalTok{) }\OperatorTok{+}\StringTok{ }
\StringTok{  }\KeywordTok{geom_segment}\NormalTok{(}\DataTypeTok{data =}\NormalTok{ S.Y2.bp.sign, }
               \KeywordTok{aes}\NormalTok{(}\DataTypeTok{x =} \DecValTok{0}\NormalTok{, }\DataTypeTok{y =} \DecValTok{0}\NormalTok{, }\DataTypeTok{xend =}\NormalTok{ x, }\DataTypeTok{yend =}\NormalTok{ y),}
               \DataTypeTok{arrow =} \KeywordTok{arrow}\NormalTok{(}\DataTypeTok{length =} \KeywordTok{unit}\NormalTok{(}\FloatTok{0.2}\NormalTok{, }\StringTok{"cm"}\NormalTok{)), }
               \DataTypeTok{color =} \StringTok{"red"}\NormalTok{, }\DataTypeTok{alpha =} \DecValTok{1}\NormalTok{, }\DataTypeTok{size =} \FloatTok{0.7}\NormalTok{) }\OperatorTok{+}
\StringTok{  }\KeywordTok{geom_text}\NormalTok{(}\DataTypeTok{data =} \KeywordTok{as.data.frame}\NormalTok{(S.Y2.text.sign), }
            \KeywordTok{aes}\NormalTok{(x, y, }\DataTypeTok{label =} \KeywordTok{rownames}\NormalTok{(S.Y2.text.sign)),}
            \DataTypeTok{color =} \StringTok{"black"}\NormalTok{) }\OperatorTok{+}
\StringTok{  }\KeywordTok{xlab}\NormalTok{(}\StringTok{"CAP1"}\NormalTok{) }\OperatorTok{+}\StringTok{ }\KeywordTok{ylab}\NormalTok{(}\StringTok{"CAP2"}\NormalTok{) }\OperatorTok{+}\StringTok{ }
\StringTok{  }\KeywordTok{ggtitle}\NormalTok{(}\KeywordTok{expression}\NormalTok{(}\KeywordTok{paste}\NormalTok{(}\StringTok{"Spatial variables and "}\NormalTok{, beta[sne]))) }\OperatorTok{+}
\StringTok{  }\KeywordTok{theme_grey}\NormalTok{() }\OperatorTok{+}
\StringTok{  }\KeywordTok{theme}\NormalTok{(}\DataTypeTok{panel.grid.minor =} \KeywordTok{element_blank}\NormalTok{(),}
        \DataTypeTok{legend.position =} \StringTok{"none"}\NormalTok{,}
        \DataTypeTok{aspect.ratio =} \FloatTok{0.8}\NormalTok{)}
\end{Highlighting}
\end{Shaded}

\subsubsection*{Analysis of the thermal variables}

As before with the spatial variable, I now do a db-RDA involving all the
thermal variables (\(E\)) followed by forward selection. There is less
explanation provided here as the reader should now be familiar with
db-RDA --- the procedure is the same as with the MEMs, just different
explanatory variables are supplied. Another difference is that the
thermal variables are not necessarily orthogonal, so I check for
collinearity using variance inflation factors (VIF).

I start with the full model and then run forward selection and repeat
the db-RDA on the reduced set. Analyses shown for \(Y1\) and \(Y2\):

\begin{Shaded}
\begin{Highlighting}[]
\CommentTok{# First Y1:}
\NormalTok{E.Y1.cs <-}\StringTok{ }\KeywordTok{capscale}\NormalTok{(Y1 }\OperatorTok{~}\NormalTok{., E1)}

\CommentTok{# Is the fit significant?}
\KeywordTok{anova}\NormalTok{(E.Y1.cs, }\DataTypeTok{parallel =} \DecValTok{4}\NormalTok{) }\CommentTok{# ... yes!}

\CommentTok{# The adjusted R2 --- the variance explained by the constrained axes:}
\NormalTok{E.Y1.R2a <-}\StringTok{ }\KeywordTok{RsquareAdj}\NormalTok{(E.Y1.cs)}\OperatorTok{$}\NormalTok{adj.r.squared}

\CommentTok{# Variance explained by full model:}
\KeywordTok{sum}\NormalTok{(E.Y1.cs}\OperatorTok{$}\NormalTok{CCA}\OperatorTok{$}\NormalTok{eig) }\OperatorTok{/}\StringTok{ }\NormalTok{E.Y1.cs}\OperatorTok{$}\NormalTok{tot.chi }\OperatorTok{*}\StringTok{ }\DecValTok{100}
\end{Highlighting}
\end{Shaded}

\begin{Shaded}
\begin{Highlighting}[]
\CommentTok{# ...and now Y2:}
\NormalTok{E.Y2.cs <-}\StringTok{ }\KeywordTok{capscale}\NormalTok{(Y2 }\OperatorTok{~}\NormalTok{., E1)}
\KeywordTok{anova}\NormalTok{(E.Y2.cs, }\DataTypeTok{parallel =} \DecValTok{4}\NormalTok{) }\CommentTok{# ... yes!}
\NormalTok{E.Y2.R2a <-}\StringTok{ }\KeywordTok{RsquareAdj}\NormalTok{(E.Y2.cs)}\OperatorTok{$}\NormalTok{adj.r.squared}
\KeywordTok{sum}\NormalTok{(E.Y2.cs}\OperatorTok{$}\NormalTok{CCA}\OperatorTok{$}\NormalTok{eig) }\OperatorTok{/}\StringTok{ }\NormalTok{E.Y2.cs}\OperatorTok{$}\NormalTok{tot.chi }\OperatorTok{*}\StringTok{ }\DecValTok{100}
\end{Highlighting}
\end{Shaded}

\begin{Shaded}
\begin{Highlighting}[]
\CommentTok{# Forward selection on Y1:}
\NormalTok{E.Y1.fwd <-}\StringTok{ }\KeywordTok{forward.sel}\NormalTok{(Y1, E1, }\DataTypeTok{adjR2thresh =}\NormalTok{ E.Y1.R2a, }\DataTypeTok{nperm =} \DecValTok{999}\NormalTok{)}

\CommentTok{# Forward selection on Y1:}
\NormalTok{E.Y2.fwd <-}\StringTok{ }\KeywordTok{forward.sel}\NormalTok{(Y2, E1, }\DataTypeTok{adjR2thresh =}\NormalTok{ E.Y2.R2a, }\DataTypeTok{nperm =} \DecValTok{999}\NormalTok{)}
\end{Highlighting}
\end{Shaded}

\begin{Shaded}
\begin{Highlighting}[]
\CommentTok{# Write the significant envs to a new object, and}
\CommentTok{# identity of significant envs in increasing order;}
\CommentTok{# first Y1:}
\NormalTok{E.Y1.sign <-}\StringTok{ }\NormalTok{E.Y1.fwd }\OperatorTok\StringTok{ }
\StringTok{  }\NormalTok{dplyr}\OperatorTok{::}\KeywordTok{select}\NormalTok{(variables) }\OperatorTok\StringTok{ }
\StringTok{  }\KeywordTok{as.vector}\NormalTok{()}

\NormalTok{E.Y1.red <-}\StringTok{ }\NormalTok{E1[, E.Y1.sign[,}\DecValTok{1}\NormalTok{]]}

\CommentTok{# Run a new env analysis on the best env variables:}
\NormalTok{E.Y1.cs2 <-}\StringTok{ }\KeywordTok{capscale}\NormalTok{(Y1 }\OperatorTok{~}\NormalTok{., E.Y1.red)}

\CommentTok{# Check for collinearity:}
\KeywordTok{vif.cca}\NormalTok{(E.Y1.cs2)}

\CommentTok{# If there are significant collinearity the collinear variables can be removed:}
\CommentTok{# E.red <- dplyr::select(E.red, -augMean)}
\CommentTok{# E.cs2 <- capscale(Y1 ~ ., E.red)}
\CommentTok{#}
\CommentTok{# check for collinearity again:}
\CommentTok{# vif.cca(E.cs2) # much better}

\CommentTok{# Test for significance:}
\KeywordTok{anova}\NormalTok{(E.Y1.cs2, }\DataTypeTok{parallel =} \DecValTok{4}\NormalTok{) }\CommentTok{# ... yes!}

\CommentTok{# Which axes are significant?}
\KeywordTok{anova}\NormalTok{(E.Y1.cs2, }\DataTypeTok{by =} \StringTok{"axis"}\NormalTok{, }\DataTypeTok{parallel =} \DecValTok{4}\NormalTok{) }\CommentTok{# ... yes!}

\CommentTok{# The significant axes:}
\NormalTok{E.Y1.axis.test <-}\StringTok{ }\KeywordTok{anova}\NormalTok{(E.Y1.cs2, }\DataTypeTok{by =} \StringTok{"terms"}\NormalTok{, }\DataTypeTok{parallel =} \DecValTok{4}\NormalTok{)}
\NormalTok{E.Y1.ax <-}\StringTok{ }\KeywordTok{which}\NormalTok{(E.Y1.axis.test[, }\DecValTok{4}\NormalTok{] }\OperatorTok{<}\StringTok{ }\FloatTok{0.05}\NormalTok{)}
\NormalTok{E.Y1.sign.ax <-}\StringTok{ }\KeywordTok{colnames}\NormalTok{(E.Y1.red[,E.Y1.ax])}

\CommentTok{# The adjusted R2 --- the variance explained by the constrained axes:}
\NormalTok{E.Y1.cs2.R2 <-}\StringTok{ }\KeywordTok{RsquareAdj}\NormalTok{(E.Y1.cs2)}\OperatorTok{$}\NormalTok{adj.r.squared}

\CommentTok{# Variance explained by reduced (final) model:}
\KeywordTok{sum}\NormalTok{(E.Y1.cs2}\OperatorTok{$}\NormalTok{CCA}\OperatorTok{$}\NormalTok{eig) }\OperatorTok{/}\StringTok{ }\NormalTok{E.Y1.cs2}\OperatorTok{$}\NormalTok{tot.chi }\OperatorTok{*}\StringTok{ }\DecValTok{100}

\CommentTok{# The biplot scores for constraining variables:}
\KeywordTok{scores}\NormalTok{(E.Y1.cs2, }\DataTypeTok{display =} \StringTok{"bp"}\NormalTok{, }\DataTypeTok{choices =} \KeywordTok{c}\NormalTok{(}\DecValTok{1}\OperatorTok{:}\DecValTok{2}\NormalTok{))}
\end{Highlighting}
\end{Shaded}

\begin{Shaded}
\begin{Highlighting}[]
\CommentTok{# ...then Y2}
\NormalTok{E.Y2.sign <-}\StringTok{ }\NormalTok{E.Y2.fwd }\OperatorTok\StringTok{ }
\StringTok{  }\NormalTok{dplyr}\OperatorTok{::}\KeywordTok{select}\NormalTok{(variables) }\OperatorTok\StringTok{ }
\StringTok{  }\KeywordTok{as.vector}\NormalTok{()}

\NormalTok{E.Y2.red <-}\StringTok{ }\NormalTok{E1[, E.Y2.sign[,}\DecValTok{1}\NormalTok{]]}

\NormalTok{E.Y2.sign <-}\StringTok{ }\KeywordTok{sort}\NormalTok{(E.Y2.fwd[, }\DecValTok{1}\NormalTok{])}
\NormalTok{E.Y2.red <-}\StringTok{ }\KeywordTok{data.frame}\NormalTok{(E1[, }\KeywordTok{c}\NormalTok{(E.Y2.sign)])}
\KeywordTok{colnames}\NormalTok{(E.Y2.red) <-}\StringTok{ }\NormalTok{E.Y2.sign}

\NormalTok{E.Y2.cs2 <-}\StringTok{ }\KeywordTok{capscale}\NormalTok{(Y2 }\OperatorTok{~}\NormalTok{., E.Y2.red)}

\KeywordTok{vif.cca}\NormalTok{(E.Y2.cs2)}
\CommentTok{# E.red <- dplyr::select(E.red, -augMean)}
\CommentTok{# E.cs2 <- capscale(Y2 ~ ., E.red)}

\CommentTok{# vif.cca(E.cs2) # much better}

\KeywordTok{anova}\NormalTok{(E.Y2.cs2, }\DataTypeTok{parallel =} \DecValTok{4}\NormalTok{) }\CommentTok{# ... yes!}

\NormalTok{E.Y2.axis.test <-}\StringTok{ }\KeywordTok{anova}\NormalTok{(E.Y2.cs2, }\DataTypeTok{by =} \StringTok{"terms"}\NormalTok{, }\DataTypeTok{parallel =} \DecValTok{4}\NormalTok{)}
\CommentTok{# E.Y2.ax <- which(E.Y2.axis.test[, 4] < 0.05) # doesn't work...}
\CommentTok{# E.Y2.sign.ax <- colnames(E.Y2.red[,E.Y2.ax])}
\NormalTok{E.Y2.sign.ax <-}\StringTok{ "annMean"} \CommentTok{# a manual cheat}

\KeywordTok{anova}\NormalTok{(E.Y2.cs2, }\DataTypeTok{by =} \StringTok{"terms"}\NormalTok{, }\DataTypeTok{parallel =} \DecValTok{4}\NormalTok{) }\CommentTok{# ... yes!}

\NormalTok{E.Y2.cs2.R2 <-}\StringTok{ }\KeywordTok{RsquareAdj}\NormalTok{(E.Y2.cs2)}\OperatorTok{$}\NormalTok{adj.r.squared}

\KeywordTok{sum}\NormalTok{(E.Y2.cs2}\OperatorTok{$}\NormalTok{CCA}\OperatorTok{$}\NormalTok{eig) }\OperatorTok{/}\StringTok{ }\NormalTok{E.Y2.cs2}\OperatorTok{$}\NormalTok{tot.chi }\OperatorTok{*}\StringTok{ }\DecValTok{100}

\KeywordTok{scores}\NormalTok{(E.Y2.cs2, }\DataTypeTok{display =} \StringTok{"bp"}\NormalTok{, }\DataTypeTok{choices =} \KeywordTok{c}\NormalTok{(}\DecValTok{1}\OperatorTok{:}\DecValTok{2}\NormalTok{))}
\end{Highlighting}
\end{Shaded}

Now I make the remaining panels of Fig. 3, these showing the spatial
arrangement associated with the site scores of the environmental
variables for \(Y1\) and \(Y2\):

\begin{Shaded}
\begin{Highlighting}[]
\CommentTok{# Plot the two significant canonical axes of the }
\CommentTok{# db-RDA with the significant MEMs. This part of Fig. 3:}
\NormalTok{E.Y1.axes <-}\StringTok{ }\KeywordTok{scores}\NormalTok{(E.Y1.cs2, }\DataTypeTok{choices =} \KeywordTok{c}\NormalTok{(}\DecValTok{1}\OperatorTok{:}\DecValTok{2}\NormalTok{),}
                   \DataTypeTok{display =} \StringTok{"lc"}\NormalTok{, }\DataTypeTok{scaling =} \DecValTok{1}\NormalTok{)}
\NormalTok{E.Y1.plt.axis1 <-}\StringTok{ }\KeywordTok{ggmap}\NormalTok{() }\OperatorTok{+}
\StringTok{  }\KeywordTok{geom_point}\NormalTok{(}\DataTypeTok{data =}\NormalTok{ sites, }\KeywordTok{aes}\NormalTok{(}\DataTypeTok{x =}\NormalTok{ Longitude, }\DataTypeTok{y =}\NormalTok{ Latitude, }\DataTypeTok{size =}\NormalTok{ E.Y1.axes[, }\DecValTok{1}\NormalTok{]),}
             \DataTypeTok{col =} \StringTok{"black"}\NormalTok{, }\DataTypeTok{shape =} \DecValTok{1}\NormalTok{) }\OperatorTok{+}
\StringTok{  }\KeywordTok{scale_size_continuous}\NormalTok{(}\DataTypeTok{guide =} \OtherTok{FALSE}\NormalTok{) }\OperatorTok{+}
\StringTok{  }\KeywordTok{ggtitle}\NormalTok{(}\KeywordTok{expression}\NormalTok{(}\KeywordTok{paste}\NormalTok{(}\StringTok{"CAP1 of thermal variables, "}\NormalTok{, beta[sim])))}

\NormalTok{E.Y1.plt.axis2 <-}\StringTok{ }\KeywordTok{ggmap}\NormalTok{() }\OperatorTok{+}
\StringTok{  }\KeywordTok{geom_point}\NormalTok{(}\DataTypeTok{data =}\NormalTok{ sites, }\KeywordTok{aes}\NormalTok{(}\DataTypeTok{x =}\NormalTok{ Longitude, }\DataTypeTok{y =}\NormalTok{ Latitude, }\DataTypeTok{size =}\NormalTok{ E.Y1.axes[, }\DecValTok{2}\NormalTok{]),}
             \DataTypeTok{col =} \StringTok{"black"}\NormalTok{, }\DataTypeTok{shape =} \DecValTok{1}\NormalTok{) }\OperatorTok{+}
\StringTok{  }\KeywordTok{scale_size_continuous}\NormalTok{(}\DataTypeTok{guide =} \OtherTok{FALSE}\NormalTok{) }\OperatorTok{+}
\StringTok{  }\KeywordTok{ggtitle}\NormalTok{(}\KeywordTok{expression}\NormalTok{(}\KeywordTok{paste}\NormalTok{(}\StringTok{"CAP2 of thermal variables, "}\NormalTok{, beta[sim])))}

\NormalTok{E.Y2.axes <-}\StringTok{ }\KeywordTok{scores}\NormalTok{(E.Y2.cs2, }\DataTypeTok{choices =} \KeywordTok{c}\NormalTok{(}\DecValTok{1}\OperatorTok{:}\DecValTok{3}\NormalTok{),}
                    \DataTypeTok{display =} \StringTok{"lc"}\NormalTok{, }\DataTypeTok{scaling =} \DecValTok{1}\NormalTok{)}

\NormalTok{E.Y2.plt.axis1 <-}\StringTok{ }\KeywordTok{ggmap}\NormalTok{() }\OperatorTok{+}
\StringTok{  }\KeywordTok{geom_point}\NormalTok{(}\DataTypeTok{data =}\NormalTok{ sites, }\KeywordTok{aes}\NormalTok{(}\DataTypeTok{x =}\NormalTok{ Longitude, }\DataTypeTok{y =}\NormalTok{ Latitude, }\DataTypeTok{size =}\NormalTok{ E.Y2.axes[, }\DecValTok{1}\NormalTok{]),}
             \DataTypeTok{col =} \StringTok{"black"}\NormalTok{, }\DataTypeTok{shape =} \DecValTok{1}\NormalTok{) }\OperatorTok{+}
\StringTok{  }\KeywordTok{scale_size_continuous}\NormalTok{(}\DataTypeTok{guide =} \OtherTok{FALSE}\NormalTok{) }\OperatorTok{+}
\StringTok{  }\KeywordTok{ggtitle}\NormalTok{(}\KeywordTok{expression}\NormalTok{(}\KeywordTok{paste}\NormalTok{(}\StringTok{"CAP1 of thermal variables, "}\NormalTok{, beta[sne])))}
\end{Highlighting}
\end{Shaded}

And now I make the remaining panels of Fig. 2 for Y1 and Y2 and the
environmental constraining vectors:

\begin{Shaded}
\begin{Highlighting}[]
\CommentTok{# The ordiplots in Fig. 2:}
\NormalTok{E.Y1.scrs <-}\StringTok{ }\KeywordTok{scores}\NormalTok{(E.Y1.cs2, }\DataTypeTok{display =} \KeywordTok{c}\NormalTok{(}\StringTok{"sp"}\NormalTok{,}\StringTok{"wa"}\NormalTok{,}\StringTok{"lc"}\NormalTok{,}\StringTok{"bp"}\NormalTok{,}\StringTok{"cn"}\NormalTok{))}
\NormalTok{E.Y1.df_sites <-}\StringTok{ }\KeywordTok{data.frame}\NormalTok{(E.Y1.scrs}\OperatorTok{$}\NormalTok{constraints)}
\NormalTok{E.Y1.df_sites}\OperatorTok{$}\NormalTok{bioreg <-}\StringTok{ }\NormalTok{bioreg}\OperatorTok{$}\NormalTok{bolton}
\NormalTok{E.Y1.df_sites}\OperatorTok{$}\NormalTok{section <-}\StringTok{ }\KeywordTok{seq}\NormalTok{(}\DecValTok{1}\OperatorTok{:}\DecValTok{58}\NormalTok{)}
\KeywordTok{colnames}\NormalTok{(E.Y1.df_sites) <-}\StringTok{ }\KeywordTok{c}\NormalTok{(}\StringTok{"x"}\NormalTok{, }\StringTok{"y"}\NormalTok{, }\StringTok{"Bioregion"}\NormalTok{, }\StringTok{"Section"}\NormalTok{)}

\NormalTok{multiplier <-}\StringTok{ }\KeywordTok{ordiArrowMul}\NormalTok{(E.Y1.scrs}\OperatorTok{$}\NormalTok{biplot)}
\NormalTok{E.Y1.bp <-}\StringTok{ }\NormalTok{E.Y1.scrs}\OperatorTok{$}\NormalTok{biplot }\OperatorTok{*}\StringTok{ }\NormalTok{multiplier}
\NormalTok{E.Y1.bp <-}\StringTok{ }\KeywordTok{as.data.frame}\NormalTok{(E.Y1.bp)}
\NormalTok{E.Y1.bp}\OperatorTok{$}\NormalTok{labels <-}\StringTok{ }\KeywordTok{rownames}\NormalTok{(E.Y1.bp)}
\KeywordTok{colnames}\NormalTok{(E.Y1.bp) <-}\StringTok{ }\KeywordTok{c}\NormalTok{(}\StringTok{"x"}\NormalTok{, }\StringTok{"y"}\NormalTok{, }\StringTok{"labels"}\NormalTok{)}
\NormalTok{E.Y1.bp.sign <-}\StringTok{ }\NormalTok{E.Y1.bp[E.Y1.bp}\OperatorTok{$}\NormalTok{labels }\OperatorTok\StringTok{ }\NormalTok{E.Y1.sign.ax,]}

\NormalTok{E.Y1.text <-}\StringTok{ }\KeywordTok{text.mult}\NormalTok{(E.Y1.scrs}\OperatorTok{$}\NormalTok{biplot)}
\NormalTok{E.Y1.text <-}\StringTok{ }\KeywordTok{as.data.frame}\NormalTok{(E.Y1.text)}
\NormalTok{E.Y1.text}\OperatorTok{$}\NormalTok{labels <-}\StringTok{ }\KeywordTok{rownames}\NormalTok{(E.Y1.text)}
\KeywordTok{colnames}\NormalTok{(E.Y1.text) <-}\StringTok{ }\KeywordTok{c}\NormalTok{(}\StringTok{"x"}\NormalTok{, }\StringTok{"y"}\NormalTok{, }\StringTok{"labels"}\NormalTok{)}
\NormalTok{E.Y1.text.sign <-}\StringTok{ }\NormalTok{E.Y1.text[E.Y1.text}\OperatorTok{$}\NormalTok{labels }\OperatorTok\StringTok{ }\NormalTok{E.Y1.sign.ax,]}

\NormalTok{E.Y1.p <-}\StringTok{ }\KeywordTok{ggplot}\NormalTok{(}\DataTypeTok{data =}\NormalTok{ E.Y1.df_sites, }\KeywordTok{aes}\NormalTok{(x, y, }\DataTypeTok{colour =}\NormalTok{ Bioregion)) }\OperatorTok{+}\StringTok{ }
\StringTok{  }\KeywordTok{geom_point}\NormalTok{(}\DataTypeTok{size =} \FloatTok{4.0}\NormalTok{) }\OperatorTok{+}\StringTok{ }
\StringTok{  }\KeywordTok{geom_text}\NormalTok{(}\KeywordTok{aes}\NormalTok{(}\DataTypeTok{label =}\NormalTok{ Section), }\DataTypeTok{size =} \FloatTok{3.0}\NormalTok{, }\DataTypeTok{col =} \StringTok{"white"}\NormalTok{) }\OperatorTok{+}\StringTok{ }
\StringTok{  }\KeywordTok{geom_segment}\NormalTok{(}\DataTypeTok{data =}\NormalTok{ E.Y1.bp.sign, }
               \KeywordTok{aes}\NormalTok{(}\DataTypeTok{x =} \DecValTok{0}\NormalTok{, }\DataTypeTok{y =} \DecValTok{0}\NormalTok{, }\DataTypeTok{xend =}\NormalTok{ x, }\DataTypeTok{yend =}\NormalTok{ y),}
               \DataTypeTok{arrow =} \KeywordTok{arrow}\NormalTok{(}\DataTypeTok{length =} \KeywordTok{unit}\NormalTok{(}\FloatTok{0.2}\NormalTok{, }\StringTok{"cm"}\NormalTok{)), }
               \DataTypeTok{color =} \StringTok{"red"}\NormalTok{, }\DataTypeTok{alpha =} \DecValTok{1}\NormalTok{, }\DataTypeTok{size =} \FloatTok{0.7}\NormalTok{) }\OperatorTok{+}
\StringTok{  }\KeywordTok{geom_text}\NormalTok{(}\DataTypeTok{data =} \KeywordTok{as.data.frame}\NormalTok{(E.Y1.text.sign), }
            \KeywordTok{aes}\NormalTok{(x, y, }\DataTypeTok{label =} \KeywordTok{rownames}\NormalTok{(E.Y1.text.sign)),}
            \DataTypeTok{color =} \StringTok{"black"}\NormalTok{) }\OperatorTok{+}
\StringTok{  }\KeywordTok{xlab}\NormalTok{(}\StringTok{"CAP1"}\NormalTok{) }\OperatorTok{+}\StringTok{ }\KeywordTok{ylab}\NormalTok{(}\StringTok{"CAP2"}\NormalTok{) }\OperatorTok{+}\StringTok{ }
\StringTok{  }\KeywordTok{ggtitle}\NormalTok{(}\KeywordTok{expression}\NormalTok{(}\KeywordTok{paste}\NormalTok{(}\StringTok{"Thermal variables and "}\NormalTok{, beta[sim]))) }\OperatorTok{+}
\StringTok{  }\KeywordTok{theme_grey}\NormalTok{() }\OperatorTok{+}
\StringTok{  }\KeywordTok{theme}\NormalTok{(}\DataTypeTok{panel.grid.minor =} \KeywordTok{element_blank}\NormalTok{(),}
        \DataTypeTok{legend.position =} \StringTok{"none"}\NormalTok{,}
        \DataTypeTok{aspect.ratio =} \FloatTok{0.8}\NormalTok{)}

\NormalTok{E.Y2.scrs <-}\StringTok{ }\KeywordTok{scores}\NormalTok{(E.Y2.cs2, }\DataTypeTok{display =} \KeywordTok{c}\NormalTok{(}\StringTok{"sp"}\NormalTok{,}\StringTok{"wa"}\NormalTok{,}\StringTok{"lc"}\NormalTok{,}\StringTok{"bp"}\NormalTok{,}\StringTok{"cn"}\NormalTok{))}
\NormalTok{E.Y2.df_sites <-}\StringTok{ }\KeywordTok{data.frame}\NormalTok{(E.Y2.scrs}\OperatorTok{$}\NormalTok{constraints)}
\NormalTok{E.Y2.df_sites}\OperatorTok{$}\NormalTok{bioreg <-}\StringTok{ }\NormalTok{bioreg}\OperatorTok{$}\NormalTok{bolton}
\NormalTok{E.Y2.df_sites}\OperatorTok{$}\NormalTok{section <-}\StringTok{ }\KeywordTok{seq}\NormalTok{(}\DecValTok{1}\OperatorTok{:}\DecValTok{58}\NormalTok{)}
\KeywordTok{colnames}\NormalTok{(E.Y2.df_sites) <-}\StringTok{ }\KeywordTok{c}\NormalTok{(}\StringTok{"x"}\NormalTok{, }\StringTok{"y"}\NormalTok{, }\StringTok{"Bioregion"}\NormalTok{, }\StringTok{"Section"}\NormalTok{)}

\NormalTok{multiplier <-}\StringTok{ }\KeywordTok{ordiArrowMul}\NormalTok{(E.Y2.scrs}\OperatorTok{$}\NormalTok{biplot, }\DataTypeTok{fill =} \FloatTok{0.45}\NormalTok{)}
\NormalTok{E.Y2.bp <-}\StringTok{ }\NormalTok{E.Y2.scrs}\OperatorTok{$}\NormalTok{biplot }\OperatorTok{*}\StringTok{ }\NormalTok{multiplier}
\NormalTok{E.Y2.bp <-}\StringTok{ }\KeywordTok{as.data.frame}\NormalTok{(E.Y2.bp)}
\NormalTok{E.Y2.bp}\OperatorTok{$}\NormalTok{labels <-}\StringTok{ }\KeywordTok{rownames}\NormalTok{(E.Y2.bp)}
\KeywordTok{colnames}\NormalTok{(E.Y2.bp) <-}\StringTok{ }\KeywordTok{c}\NormalTok{(}\StringTok{"x"}\NormalTok{, }\StringTok{"y"}\NormalTok{, }\StringTok{"labels"}\NormalTok{)}
\NormalTok{E.Y2.bp.sign <-}\StringTok{ }\NormalTok{E.Y2.bp[E.Y2.bp}\OperatorTok{$}\NormalTok{labels }\OperatorTok\StringTok{ }\NormalTok{E.Y2.sign.ax,]}

\NormalTok{E.Y2.text <-}\StringTok{ }\KeywordTok{text.mult}\NormalTok{(E.Y2.scrs}\OperatorTok{$}\NormalTok{biplot, }\DataTypeTok{fill =} \FloatTok{0.45}\NormalTok{)}
\NormalTok{E.Y2.text <-}\StringTok{ }\KeywordTok{as.data.frame}\NormalTok{(E.Y2.text)}
\NormalTok{E.Y2.text}\OperatorTok{$}\NormalTok{labels <-}\StringTok{ }\KeywordTok{rownames}\NormalTok{(E.Y2.text)}
\KeywordTok{colnames}\NormalTok{(E.Y2.text) <-}\StringTok{ }\KeywordTok{c}\NormalTok{(}\StringTok{"x"}\NormalTok{, }\StringTok{"y"}\NormalTok{, }\StringTok{"labels"}\NormalTok{)}
\NormalTok{E.Y2.text.sign <-}\StringTok{ }\NormalTok{E.Y2.text[E.Y2.text}\OperatorTok{$}\NormalTok{labels }\OperatorTok\StringTok{ }\NormalTok{E.Y2.sign.ax,]}

\NormalTok{E.Y2.p <-}\StringTok{ }\KeywordTok{ggplot}\NormalTok{(}\DataTypeTok{data =}\NormalTok{ E.Y2.df_sites, }\KeywordTok{aes}\NormalTok{(x, y, }\DataTypeTok{colour =}\NormalTok{ Bioregion)) }\OperatorTok{+}\StringTok{ }
\StringTok{  }\KeywordTok{geom_point}\NormalTok{(}\DataTypeTok{size =} \FloatTok{4.0}\NormalTok{) }\OperatorTok{+}\StringTok{ }
\StringTok{  }\KeywordTok{geom_text}\NormalTok{(}\KeywordTok{aes}\NormalTok{(}\DataTypeTok{label =}\NormalTok{ Section), }\DataTypeTok{size =} \FloatTok{3.0}\NormalTok{, }\DataTypeTok{col =} \StringTok{"white"}\NormalTok{) }\OperatorTok{+}\StringTok{ }
\StringTok{  }\KeywordTok{geom_segment}\NormalTok{(}\DataTypeTok{data =}\NormalTok{ E.Y2.bp.sign, }
               \KeywordTok{aes}\NormalTok{(}\DataTypeTok{x =} \DecValTok{0}\NormalTok{, }\DataTypeTok{y =} \DecValTok{0}\NormalTok{, }\DataTypeTok{xend =}\NormalTok{ x, }\DataTypeTok{yend =}\NormalTok{ y),}
               \DataTypeTok{arrow =} \KeywordTok{arrow}\NormalTok{(}\DataTypeTok{length =} \KeywordTok{unit}\NormalTok{(}\FloatTok{0.2}\NormalTok{, }\StringTok{"cm"}\NormalTok{)), }
               \DataTypeTok{color =} \StringTok{"red"}\NormalTok{, }\DataTypeTok{alpha =} \DecValTok{1}\NormalTok{, }\DataTypeTok{size =} \FloatTok{0.7}\NormalTok{) }\OperatorTok{+}
\StringTok{  }\KeywordTok{geom_text}\NormalTok{(}\DataTypeTok{data =} \KeywordTok{as.data.frame}\NormalTok{(E.Y2.text.sign), }
            \KeywordTok{aes}\NormalTok{(x, y, }\DataTypeTok{label =} \KeywordTok{rownames}\NormalTok{(E.Y2.text.sign)),}
            \DataTypeTok{color =} \StringTok{"black"}\NormalTok{) }\OperatorTok{+}
\StringTok{  }\KeywordTok{xlab}\NormalTok{(}\StringTok{"CAP1"}\NormalTok{) }\OperatorTok{+}\StringTok{ }\KeywordTok{ylab}\NormalTok{(}\StringTok{"CAP2"}\NormalTok{) }\OperatorTok{+}\StringTok{ }
\StringTok{  }\KeywordTok{ggtitle}\NormalTok{(}\KeywordTok{expression}\NormalTok{(}\KeywordTok{paste}\NormalTok{(}\StringTok{"Thermal variables and "}\NormalTok{, beta[sne]))) }\OperatorTok{+}
\StringTok{  }\KeywordTok{theme_grey}\NormalTok{() }\OperatorTok{+}
\StringTok{  }\KeywordTok{theme}\NormalTok{(}\DataTypeTok{panel.grid.minor =} \KeywordTok{element_blank}\NormalTok{(),}
        \DataTypeTok{legend.position =} \KeywordTok{c}\NormalTok{(.}\DecValTok{80}\NormalTok{, .}\DecValTok{75}\NormalTok{),}
        \DataTypeTok{aspect.ratio =} \FloatTok{0.8}\NormalTok{)}
\end{Highlighting}
\end{Shaded}

Here I now assemble the various panels into what we see produced in Fig.
2 in the paper:

\begin{Shaded}
\begin{Highlighting}[]
\KeywordTok{pdf}\NormalTok{(}\StringTok{"Fig2.pdf"}\NormalTok{, }\DataTypeTok{width =} \DecValTok{9}\NormalTok{, }\DataTypeTok{height =} \DecValTok{8}\NormalTok{)}
\NormalTok{grid}\OperatorTok{::}\KeywordTok{grid.newpage}\NormalTok{()}
\NormalTok{grid}\OperatorTok{::}\KeywordTok{pushViewport}\NormalTok{(grid}\OperatorTok{::}\KeywordTok{viewport}\NormalTok{(}\DataTypeTok{layout =}\NormalTok{ grid}\OperatorTok{::}\KeywordTok{grid.layout}\NormalTok{(}\DecValTok{2}\NormalTok{,}\DecValTok{2}\NormalTok{)))}
\NormalTok{vplayout <-}\StringTok{ }\ControlFlowTok{function}\NormalTok{(x, y) grid}\OperatorTok{::}\KeywordTok{viewport}\NormalTok{(}\DataTypeTok{layout.pos.row =}\NormalTok{ x, }\DataTypeTok{layout.pos.col =}\NormalTok{ y)}
\KeywordTok{print}\NormalTok{(E.Y1.p, }\DataTypeTok{vp =} \KeywordTok{vplayout}\NormalTok{(}\DecValTok{1}\NormalTok{,}\DecValTok{1}\NormalTok{))}
\KeywordTok{print}\NormalTok{(E.Y2.p, }\DataTypeTok{vp =} \KeywordTok{vplayout}\NormalTok{(}\DecValTok{1}\NormalTok{,}\DecValTok{2}\NormalTok{))}
\KeywordTok{print}\NormalTok{(S.Y1.p, }\DataTypeTok{vp =} \KeywordTok{vplayout}\NormalTok{(}\DecValTok{2}\NormalTok{,}\DecValTok{1}\NormalTok{))}
\KeywordTok{print}\NormalTok{(S.Y2.p, }\DataTypeTok{vp =} \KeywordTok{vplayout}\NormalTok{(}\DecValTok{2}\NormalTok{,}\DecValTok{2}\NormalTok{))}
\KeywordTok{dev.off}\NormalTok{()}
\end{Highlighting}
\end{Shaded}

And I do the same with assembling the panels that form Fig. 3 in the
paper:

\begin{Shaded}
\begin{Highlighting}[]
\KeywordTok{pdf}\NormalTok{(}\StringTok{"Fig3.pdf"}\NormalTok{, }\DataTypeTok{width =} \DecValTok{9}\NormalTok{, }\DataTypeTok{height =} \DecValTok{7}\NormalTok{)}
\NormalTok{grid}\OperatorTok{::}\KeywordTok{grid.newpage}\NormalTok{()}
\NormalTok{grid}\OperatorTok{::}\KeywordTok{pushViewport}\NormalTok{(grid}\OperatorTok{::}\KeywordTok{viewport}\NormalTok{(}\DataTypeTok{layout =}\NormalTok{ grid}\OperatorTok{::}\KeywordTok{grid.layout}\NormalTok{(}\DecValTok{3}\NormalTok{,}\DecValTok{2}\NormalTok{)))}
\NormalTok{vplayout <-}\StringTok{ }\ControlFlowTok{function}\NormalTok{(x, y) grid}\OperatorTok{::}\KeywordTok{viewport}\NormalTok{(}\DataTypeTok{layout.pos.row =}\NormalTok{ x, }\DataTypeTok{layout.pos.col =}\NormalTok{ y)}
\KeywordTok{print}\NormalTok{(E.Y1.plt.axis1, }\DataTypeTok{vp =} \KeywordTok{vplayout}\NormalTok{(}\DecValTok{1}\NormalTok{,}\DecValTok{1}\NormalTok{))}
\KeywordTok{print}\NormalTok{(E.Y1.plt.axis2, }\DataTypeTok{vp =} \KeywordTok{vplayout}\NormalTok{(}\DecValTok{1}\NormalTok{,}\DecValTok{2}\NormalTok{))}
\KeywordTok{print}\NormalTok{(E.Y2.plt.axis1, }\DataTypeTok{vp =} \KeywordTok{vplayout}\NormalTok{(}\DecValTok{2}\NormalTok{,}\DecValTok{1}\NormalTok{))}
\KeywordTok{print}\NormalTok{(S.Y1.plt.axis1, }\DataTypeTok{vp =} \KeywordTok{vplayout}\NormalTok{(}\DecValTok{3}\NormalTok{,}\DecValTok{1}\NormalTok{))}
\KeywordTok{print}\NormalTok{(S.Y2.plt.axis1, }\DataTypeTok{vp =} \KeywordTok{vplayout}\NormalTok{(}\DecValTok{3}\NormalTok{,}\DecValTok{2}\NormalTok{))}
\KeywordTok{dev.off}\NormalTok{()}
\end{Highlighting}
\end{Shaded}

\subsubsection*{Partitioning of variance}

Lastly, using \textbf{vegan}'s \texttt{varpart()} function, I partition
the variance between the MEM variables and the thermal variables
(Peres-Neto and Legendre, 2010; Peres-Neto et al., 2006).

\begin{Shaded}
\begin{Highlighting}[]
\CommentTok{# These lines of code produce a few figures to visually understand}
\CommentTok{# the variance partitioning on Y1:}
\NormalTok{vp2.Y1 <-}\StringTok{ }\KeywordTok{varpart}\NormalTok{(Y1, E.Y1.red, S.Y1.red)}
\KeywordTok{par}\NormalTok{(}\DataTypeTok{mfrow =} \KeywordTok{c}\NormalTok{(}\DecValTok{1}\NormalTok{, }\DecValTok{2}\NormalTok{))}
\KeywordTok{showvarparts}\NormalTok{(}\DecValTok{2}\NormalTok{, }\KeywordTok{c}\NormalTok{(}\StringTok{"Environ-}\CharTok{\textbackslash{}n}\StringTok{ment"}\NormalTok{,}\StringTok{""}\NormalTok{,}\StringTok{"Spatial"}\NormalTok{,}\StringTok{""}\NormalTok{))}
\KeywordTok{plot}\NormalTok{(vp2.Y1, }\DataTypeTok{digits =} \DecValTok{2}\NormalTok{)}
\KeywordTok{par}\NormalTok{(}\DataTypeTok{mfrow =} \KeywordTok{c}\NormalTok{(}\DecValTok{1}\NormalTok{, }\DecValTok{1}\NormalTok{))}

\CommentTok{# Now I test the significant fractions [a], [b] and [c]...}
\NormalTok{ES.Y1.red <-}\StringTok{ }\KeywordTok{cbind}\NormalTok{(E.Y1.red, S.Y1.red)}

\CommentTok{# Fraction E | S; pure environmental, i.e. [a]:}
\KeywordTok{anova.cca}\NormalTok{(}\KeywordTok{capscale}\NormalTok{(Y1 }\OperatorTok{~}\StringTok{ }\NormalTok{augMean }\OperatorTok{+}\StringTok{ }\NormalTok{febRange }\OperatorTok{+}\StringTok{ }\NormalTok{febSD }\OperatorTok{+}\StringTok{ }\NormalTok{augSD }\OperatorTok{+}
\StringTok{                  }\KeywordTok{Condition}\NormalTok{(MEM1 }\OperatorTok{+}\StringTok{ }\NormalTok{MEM2 }\OperatorTok{+}\StringTok{ }\NormalTok{MEM3 }\OperatorTok{+}\StringTok{ }\NormalTok{MEM4 }\OperatorTok{+}\StringTok{ }\NormalTok{MEM5 }\OperatorTok{+}
\StringTok{                              }\NormalTok{MEM6 }\OperatorTok{+}\StringTok{ }\NormalTok{MEM7 }\OperatorTok{+}\StringTok{ }\NormalTok{MEM8 }\OperatorTok{+}\StringTok{ }\NormalTok{MEM9 }\OperatorTok{+}\StringTok{ }\NormalTok{MEM10 }\OperatorTok{+}
\StringTok{                              }\NormalTok{MEM13 }\OperatorTok{+}\StringTok{ }\NormalTok{MEM15 }\OperatorTok{+}\StringTok{ }\NormalTok{MEM16 }\OperatorTok{+}
\StringTok{                              }\NormalTok{MEM18 }\OperatorTok{+}\StringTok{ }\NormalTok{MEM19 }\OperatorTok{+}\StringTok{ }\NormalTok{MEM20),}
                \DataTypeTok{data =}\NormalTok{ ES.Y1.red), }\DataTypeTok{parallel =} \DecValTok{4}\NormalTok{, }\DataTypeTok{step =} \DecValTok{1000}\NormalTok{)}

\CommentTok{# Fraction S | E; pure spatial, i.e. [c]:}
\KeywordTok{anova.cca}\NormalTok{(}\KeywordTok{capscale}\NormalTok{(Y1 }\OperatorTok{~}\StringTok{ }\NormalTok{MEM1 }\OperatorTok{+}\StringTok{ }\NormalTok{MEM2 }\OperatorTok{+}\StringTok{ }\NormalTok{MEM3 }\OperatorTok{+}\StringTok{ }\NormalTok{MEM4 }\OperatorTok{+}\StringTok{ }\NormalTok{MEM5 }\OperatorTok{+}
\StringTok{                              }\NormalTok{MEM6 }\OperatorTok{+}\StringTok{ }\NormalTok{MEM7 }\OperatorTok{+}\StringTok{ }\NormalTok{MEM8 }\OperatorTok{+}\StringTok{ }\NormalTok{MEM9 }\OperatorTok{+}\StringTok{ }\NormalTok{MEM10 }\OperatorTok{+}
\StringTok{                              }\NormalTok{MEM13 }\OperatorTok{+}\StringTok{ }\NormalTok{MEM15 }\OperatorTok{+}\StringTok{ }\NormalTok{MEM16 }\OperatorTok{+}
\StringTok{                              }\NormalTok{MEM18 }\OperatorTok{+}\StringTok{ }\NormalTok{MEM19 }\OperatorTok{+}\StringTok{ }\NormalTok{MEM20 }\OperatorTok{+}
\StringTok{                  }\KeywordTok{Condition}\NormalTok{(augMean }\OperatorTok{+}\StringTok{ }\NormalTok{febRange }\OperatorTok{+}\StringTok{ }\NormalTok{febSD }\OperatorTok{+}\StringTok{ }\NormalTok{augSD),}
                \DataTypeTok{data =}\NormalTok{ ES.Y1.red), }\DataTypeTok{parallel =} \DecValTok{4}\NormalTok{, }\DataTypeTok{step =} \DecValTok{1000}\NormalTok{)}

\CommentTok{# Fraction E; environmental, i.e. [a] + [b]:}
\KeywordTok{anova.cca}\NormalTok{(}\KeywordTok{capscale}\NormalTok{(Y1 }\OperatorTok{~}\NormalTok{., E.Y1.red), }\DataTypeTok{parallel =} \DecValTok{4}\NormalTok{, }\DataTypeTok{step =} \DecValTok{1000}\NormalTok{)}

\CommentTok{# Fractions S; spatial, i.e. [b] + [c]:}
\KeywordTok{anova.cca}\NormalTok{(}\KeywordTok{capscale}\NormalTok{(Y1 }\OperatorTok{~}\NormalTok{., S.Y1.red), }\DataTypeTok{parallel =} \DecValTok{4}\NormalTok{, }\DataTypeTok{step =} \DecValTok{1000}\NormalTok{)}

\CommentTok{# Fractions E + S; spatial and environmental, i.e. [a] + [b] + [c]:}
\KeywordTok{anova.cca}\NormalTok{(}\KeywordTok{capscale}\NormalTok{(Y1 }\OperatorTok{~}\NormalTok{., }\KeywordTok{cbind}\NormalTok{(E.Y1.red, S.Y1.red)), }\DataTypeTok{parallel =} \DecValTok{4}\NormalTok{, }\DataTypeTok{step =} \DecValTok{1000}\NormalTok{)}

\CommentTok{# And now the partitioning of the variance in Y2:}
\NormalTok{(vp2.Y2 <-}\StringTok{ }\KeywordTok{varpart}\NormalTok{(Y2, E.Y2.red, S.Y2.red))}
\KeywordTok{par}\NormalTok{(}\DataTypeTok{mfrow =} \KeywordTok{c}\NormalTok{(}\DecValTok{1}\NormalTok{, }\DecValTok{2}\NormalTok{))}
\KeywordTok{showvarparts}\NormalTok{(}\DecValTok{2}\NormalTok{, }\KeywordTok{c}\NormalTok{(}\StringTok{"Environ-}\CharTok{\textbackslash{}n}\StringTok{ment"}\NormalTok{,}\StringTok{""}\NormalTok{,}\StringTok{"Spatial"}\NormalTok{,}\StringTok{""}\NormalTok{))}
\KeywordTok{plot}\NormalTok{(vp2.Y2, }\DataTypeTok{digits =} \DecValTok{2}\NormalTok{)}
\KeywordTok{par}\NormalTok{(}\DataTypeTok{mfrow =} \KeywordTok{c}\NormalTok{(}\DecValTok{1}\NormalTok{, }\DecValTok{1}\NormalTok{))}

\CommentTok{# Tests the significant fractions [a], [b] and [c]...}
\NormalTok{ES.Y2.red <-}\StringTok{ }\KeywordTok{cbind}\NormalTok{(E.Y2.red, S.Y2.red)}

\CommentTok{# Fraction E | S; pure environmental, i.e. [a]:}
\KeywordTok{anova.cca}\NormalTok{(}\KeywordTok{capscale}\NormalTok{(Y2 }\OperatorTok{~}\StringTok{ }\NormalTok{annMean }\OperatorTok{+}
\StringTok{                  }\KeywordTok{Condition}\NormalTok{(MEM1 }\OperatorTok{+}\StringTok{ }\NormalTok{MEM2 }\OperatorTok{+}\StringTok{ }\NormalTok{MEM3 }\OperatorTok{+}\StringTok{ }\NormalTok{MEM5),}
                \DataTypeTok{data =}\NormalTok{ ES.Y2.red), }\DataTypeTok{parallel =} \DecValTok{4}\NormalTok{, }\DataTypeTok{step =} \DecValTok{1000}\NormalTok{)}

\CommentTok{# Fraction S | E; pure spatial, i.e. [c]:}
\KeywordTok{anova.cca}\NormalTok{(}\KeywordTok{capscale}\NormalTok{(Y2 }\OperatorTok{~}\StringTok{ }\NormalTok{MEM1 }\OperatorTok{+}\StringTok{ }\NormalTok{MEM2 }\OperatorTok{+}\StringTok{ }\NormalTok{MEM3 }\OperatorTok{+}\StringTok{ }\NormalTok{MEM5 }\OperatorTok{+}
\StringTok{                  }\KeywordTok{Condition}\NormalTok{(annMean),}
                \DataTypeTok{data =}\NormalTok{ ES.Y2.red), }\DataTypeTok{parallel =} \DecValTok{4}\NormalTok{, }\DataTypeTok{step =} \DecValTok{1000}\NormalTok{)}

\CommentTok{# Fraction E; environmental, i.e. [a] + [b]:}
\KeywordTok{anova.cca}\NormalTok{(}\KeywordTok{capscale}\NormalTok{(Y2 }\OperatorTok{~}\NormalTok{., E.Y2.red), }\DataTypeTok{parallel =} \DecValTok{4}\NormalTok{, }\DataTypeTok{step =} \DecValTok{1000}\NormalTok{)}

\CommentTok{# Fractions S; spatial, i.e. [b] + [c]:}
\KeywordTok{anova.cca}\NormalTok{(}\KeywordTok{capscale}\NormalTok{(Y2 }\OperatorTok{~}\NormalTok{., S.Y2.red), }\DataTypeTok{parallel =} \DecValTok{4}\NormalTok{, }\DataTypeTok{step =} \DecValTok{1000}\NormalTok{)}

\CommentTok{# Fractions E + S; spatial and environmental, i.e. [a] + [b] + [c]:}
\KeywordTok{anova.cca}\NormalTok{(}\KeywordTok{capscale}\NormalTok{(Y2 }\OperatorTok{~}\NormalTok{., }\KeywordTok{cbind}\NormalTok{(E.Y2.red, S.Y2.red)), }\DataTypeTok{parallel =} \DecValTok{4}\NormalTok{, }\DataTypeTok{step =} \DecValTok{1000}\NormalTok{)}
\end{Highlighting}
\end{Shaded}

\newpage

\section*{Appendix C}

\subsection*{Network graph of \textbeta-diversity}

I delved deeper into the patterns of \textbeta-diversity by examining
the properties of the full dissimilarity matrix, which gives regional
\textbeta-diversity mentioned above. This matrix describes all pairwise
combinations of sections (582 -- 1 = 3363), and as such gives us a
regional perspective (Anderson et al., 2013). The usual visualisation
approach is to plot the dissimilarity metric as a function of
geographical distance along the gradient or with respect to the distance
between corresponding pairs of sections (Davidar et al., 2007;
\emph{e.g.} Nekola et al., 1999); these visualisations are provided
here. The plots of dissimilarities were colour-coded according to the
bioregion to which the section pairs belong (the Benguela Marine
Province (BMP; \textbf{1}--\textbf{17}), the Benguela-Agulhas Transition
Zone (B-ATZ; \textbf{18}--\textbf{22}), the Agulhas Marine Province
(AMP; \textbf{19}--\textbf{43}/\textbf{44} --- the location of this
transition is somewhat uncertain at this stage) and the East Coast
Transition Zone (ECTZ; \textbf{44}/\textbf{45}--\textbf{58})
(\emph{sensu} Bolton and Anderson, 2004) to distinguish bioregional
properties of species distribution from the wider geographical scale
structure along the whole coastline. In doing so, the change in
\textbeta-diversity per unit of separating distance between sections
(km\textsuperscript{-1}) could be calculated for each bioregion using
linear regression. Since the connectivity between sections is
constrained by their location along the shore, I calculated the
distances between sections not as `as the crow flies' distances
(\emph{e.g.} Section \textbf{1} is not connected in a straight line to
Section \textbf{58} because of the intervening land in-between), but as
the great circle geodesic distances between each pair of sections along
a network of connected sections (vertices on a network graph). In other
words, travelling from Section \textbf{1} to Section \textbf{58}
requires travelling first along the coast through Section \textbf{2},
then Section \textbf{3}, and eventually all the way up to Section
\textbf{58}. The total distance between a pair of arbitrary sections is
therefore the cumulative sum of the great circle distances between each
consecutive pair of intervening sections along the `route'. This
information is encapsulated as a square geodesic distance matrix, and
can supply the distance along the abscissa against which species
dissimilarities are plotted along the ordinate. The plots showing the
relationship between \textbeta-diversity with distance are limited
because they do not provide a geographical context. To overcome this
problem, I relied on a visualisation technique not commonly found in
biogeographical studies to explicitly provide the geographical context.
I structured the sections as vertices of a network graph and assigned to
them their geographical coordinates to force a familiar layout of the
graph --- when plotted on geographic coordinates, the sections form a
map of South Africa. The species dissimilarities were assigned as edge
weights (the lines connecting the \textbf{58} coastal sections) between
pairs of sections, and added to the map. The weights are directly
proportional to the thickness of the edges, and colours assigned to
vertices (points, or the 58 coastal sections) cluster the sections into
their bioregions. Initially I used the \textbf{igraph} package that many
people rave about, but I found it bothersome. So I devised a cunning way
to create network graphs from scratch with some \textbf{dplyr} and
\textbf{ggplot2} magick. I suppose that if I really wanted to I could
have made neat functions here (and elsewhere) to reduce some of the
repetitive nature of my code, but I really coudn't be bother doing that.

\begin{Shaded}
\begin{Highlighting}[]
\CommentTok{# Visualise the pairwise dissimilarities as network graphs where the }
\CommentTok{# vertices are geographical coordinates and the edge lengths are the geodesic }
\CommentTok{# distances. }
\CommentTok{# These visualisations appear in the paper as Fig. 4.}
\KeywordTok{colnames}\NormalTok{(sites) <-}\StringTok{ }\KeywordTok{c}\NormalTok{(}\StringTok{"lat"}\NormalTok{, }\StringTok{"lon"}\NormalTok{)}
\NormalTok{sites <-}\StringTok{ }\KeywordTok{cbind}\NormalTok{(}\KeywordTok{data.frame}\NormalTok{(}\DataTypeTok{site =} \KeywordTok{seq}\NormalTok{(}\DecValTok{1}\OperatorTok{:}\DecValTok{58}\NormalTok{)), sites)}

\NormalTok{Y1.sl <-}\StringTok{ }\KeywordTok{as.data.frame}\NormalTok{(}\KeywordTok{expand.grid}\NormalTok{(}\KeywordTok{seq}\NormalTok{(}\DecValTok{1}\OperatorTok{:}\DecValTok{58}\NormalTok{), }\KeywordTok{seq}\NormalTok{(}\DecValTok{1}\OperatorTok{:}\DecValTok{58}\NormalTok{)))}
\KeywordTok{colnames}\NormalTok{(Y1.sl) <-}\StringTok{ }\KeywordTok{c}\NormalTok{(}\StringTok{"to"}\NormalTok{, }\StringTok{"from"}\NormalTok{)}

\NormalTok{Y2.sl <-}\StringTok{ }\NormalTok{Y1.sl}

\CommentTok{# First Y1:}
\NormalTok{Y1.sl}\OperatorTok{$}\NormalTok{Y1 <-}\StringTok{ }\KeywordTok{as.vector}\NormalTok{(Y1)}

\NormalTok{Y1.sl.BMP <-}\StringTok{ }\NormalTok{Y1.sl }\OperatorTok
\StringTok{  }\NormalTok{dplyr}\OperatorTok{::}\KeywordTok{left_join}\NormalTok{(., sites, }\DataTypeTok{by =} \KeywordTok{c}\NormalTok{(}\StringTok{"to"}\NormalTok{ =}\StringTok{ "site"}\NormalTok{)) }\OperatorTok\StringTok{ }
\StringTok{  }\NormalTok{dplyr}\OperatorTok{::}\KeywordTok{left_join}\NormalTok{(., sites, }\DataTypeTok{by =} \KeywordTok{c}\NormalTok{(}\StringTok{"from"}\NormalTok{ =}\StringTok{ "site"}\NormalTok{)) }\OperatorTok\StringTok{ }
\StringTok{  }\NormalTok{dplyr}\OperatorTok{::}\KeywordTok{filter}\NormalTok{(Y1 }\OperatorTok{<=}\StringTok{ }\FloatTok{0.5} \OperatorTok{&}\StringTok{ }\NormalTok{Y1 }\OperatorTok{!=}\StringTok{ }\DecValTok{0}\NormalTok{) }\OperatorTok\StringTok{ }
\StringTok{  }\NormalTok{dplyr}\OperatorTok{::}\KeywordTok{filter}\NormalTok{(from }\OperatorTok{!=}\StringTok{ }\NormalTok{to }\OperatorTok{&}\StringTok{ }\NormalTok{from }\OperatorTok{<=}\StringTok{ }\DecValTok{16}\NormalTok{)}

\NormalTok{Y1.sl.BATZ <-}\StringTok{ }\NormalTok{Y1.sl }\OperatorTok
\StringTok{  }\NormalTok{dplyr}\OperatorTok{::}\KeywordTok{left_join}\NormalTok{(., sites, }\DataTypeTok{by =} \KeywordTok{c}\NormalTok{(}\StringTok{"to"}\NormalTok{ =}\StringTok{ "site"}\NormalTok{)) }\OperatorTok\StringTok{ }
\StringTok{  }\NormalTok{dplyr}\OperatorTok{::}\KeywordTok{left_join}\NormalTok{(., sites, }\DataTypeTok{by =} \KeywordTok{c}\NormalTok{(}\StringTok{"from"}\NormalTok{ =}\StringTok{ "site"}\NormalTok{)) }\OperatorTok\StringTok{ }
\StringTok{  }\NormalTok{dplyr}\OperatorTok{::}\KeywordTok{filter}\NormalTok{(Y1 }\OperatorTok{<=}\StringTok{ }\FloatTok{0.5} \OperatorTok{&}\StringTok{ }\NormalTok{Y1 }\OperatorTok{!=}\StringTok{ }\DecValTok{0}\NormalTok{) }\OperatorTok\StringTok{ }
\StringTok{  }\NormalTok{dplyr}\OperatorTok{::}\KeywordTok{filter}\NormalTok{(from }\OperatorTok{!=}\StringTok{ }\NormalTok{to }\OperatorTok{&}\StringTok{ }\NormalTok{from }\OperatorTok{>}\StringTok{ }\DecValTok{16} \OperatorTok{&}\StringTok{ }\NormalTok{from }\OperatorTok{<=}\StringTok{ }\DecValTok{21}\NormalTok{)}

\NormalTok{Y1.sl.AMP <-}\StringTok{ }\NormalTok{Y1.sl }\OperatorTok
\StringTok{  }\NormalTok{dplyr}\OperatorTok{::}\KeywordTok{left_join}\NormalTok{(., sites, }\DataTypeTok{by =} \KeywordTok{c}\NormalTok{(}\StringTok{"to"}\NormalTok{ =}\StringTok{ "site"}\NormalTok{)) }\OperatorTok\StringTok{ }
\StringTok{  }\NormalTok{dplyr}\OperatorTok{::}\KeywordTok{left_join}\NormalTok{(., sites, }\DataTypeTok{by =} \KeywordTok{c}\NormalTok{(}\StringTok{"from"}\NormalTok{ =}\StringTok{ "site"}\NormalTok{)) }\OperatorTok\StringTok{ }
\StringTok{  }\NormalTok{dplyr}\OperatorTok{::}\KeywordTok{filter}\NormalTok{(Y1 }\OperatorTok{<=}\StringTok{ }\FloatTok{0.5} \OperatorTok{&}\StringTok{ }\NormalTok{Y1 }\OperatorTok{!=}\StringTok{ }\DecValTok{0}\NormalTok{) }\OperatorTok\StringTok{ }
\StringTok{  }\NormalTok{dplyr}\OperatorTok{::}\KeywordTok{filter}\NormalTok{(from }\OperatorTok{!=}\StringTok{ }\NormalTok{to }\OperatorTok{&}\StringTok{ }\NormalTok{from }\OperatorTok{>}\StringTok{ }\DecValTok{21} \OperatorTok{&}\StringTok{ }\NormalTok{from }\OperatorTok{<=}\StringTok{ }\DecValTok{41}\NormalTok{)}

\NormalTok{Y1.sl.ECTZ <-}\StringTok{ }\NormalTok{Y1.sl }\OperatorTok
\StringTok{  }\NormalTok{dplyr}\OperatorTok{::}\KeywordTok{left_join}\NormalTok{(., sites, }\DataTypeTok{by =} \KeywordTok{c}\NormalTok{(}\StringTok{"to"}\NormalTok{ =}\StringTok{ "site"}\NormalTok{)) }\OperatorTok\StringTok{ }
\StringTok{  }\NormalTok{dplyr}\OperatorTok{::}\KeywordTok{left_join}\NormalTok{(., sites, }\DataTypeTok{by =} \KeywordTok{c}\NormalTok{(}\StringTok{"from"}\NormalTok{ =}\StringTok{ "site"}\NormalTok{)) }\OperatorTok\StringTok{ }
\StringTok{  }\NormalTok{dplyr}\OperatorTok{::}\KeywordTok{filter}\NormalTok{(Y1 }\OperatorTok{<=}\StringTok{ }\FloatTok{0.5} \OperatorTok{&}\StringTok{ }\NormalTok{Y1 }\OperatorTok{!=}\StringTok{ }\DecValTok{0}\NormalTok{) }\OperatorTok\StringTok{ }
\StringTok{  }\NormalTok{dplyr}\OperatorTok{::}\KeywordTok{filter}\NormalTok{(from }\OperatorTok{!=}\StringTok{ }\NormalTok{to }\OperatorTok{&}\StringTok{ }\NormalTok{from }\OperatorTok{>}\StringTok{ }\DecValTok{41}\NormalTok{)}

\NormalTok{Y1.sl <-}\StringTok{ }\KeywordTok{rbind}\NormalTok{(Y1.sl.BMP, Y1.sl.BATZ, Y1.sl.AMP, Y1.sl.ECTZ)}

\CommentTok{# and then Y2:}
\NormalTok{Y2.sl}\OperatorTok{$}\NormalTok{Y2 <-}\StringTok{ }\KeywordTok{as.vector}\NormalTok{(Y2)}

\NormalTok{Y2.sl.BMP <-}\StringTok{ }\NormalTok{Y2.sl }\OperatorTok
\StringTok{  }\NormalTok{dplyr}\OperatorTok{::}\KeywordTok{left_join}\NormalTok{(., sites, }\DataTypeTok{by =} \KeywordTok{c}\NormalTok{(}\StringTok{"to"}\NormalTok{ =}\StringTok{ "site"}\NormalTok{)) }\OperatorTok\StringTok{ }
\StringTok{  }\NormalTok{dplyr}\OperatorTok{::}\KeywordTok{left_join}\NormalTok{(., sites, }\DataTypeTok{by =} \KeywordTok{c}\NormalTok{(}\StringTok{"from"}\NormalTok{ =}\StringTok{ "site"}\NormalTok{)) }\OperatorTok\StringTok{ }
\StringTok{  }\NormalTok{dplyr}\OperatorTok{::}\KeywordTok{filter}\NormalTok{(Y2 }\OperatorTok{<=}\StringTok{ }\FloatTok{0.5} \OperatorTok{&}\StringTok{ }\NormalTok{Y2 }\OperatorTok{!=}\StringTok{ }\DecValTok{0}\NormalTok{) }\OperatorTok\StringTok{ }
\StringTok{  }\NormalTok{dplyr}\OperatorTok{::}\KeywordTok{filter}\NormalTok{(from }\OperatorTok{!=}\StringTok{ }\NormalTok{to }\OperatorTok{&}\StringTok{ }\NormalTok{from }\OperatorTok{<=}\StringTok{ }\DecValTok{16}\NormalTok{)}

\NormalTok{Y2.sl.BATZ <-}\StringTok{ }\NormalTok{Y2.sl }\OperatorTok
\StringTok{  }\NormalTok{dplyr}\OperatorTok{::}\KeywordTok{left_join}\NormalTok{(., sites, }\DataTypeTok{by =} \KeywordTok{c}\NormalTok{(}\StringTok{"to"}\NormalTok{ =}\StringTok{ "site"}\NormalTok{)) }\OperatorTok\StringTok{ }
\StringTok{  }\NormalTok{dplyr}\OperatorTok{::}\KeywordTok{left_join}\NormalTok{(., sites, }\DataTypeTok{by =} \KeywordTok{c}\NormalTok{(}\StringTok{"from"}\NormalTok{ =}\StringTok{ "site"}\NormalTok{)) }\OperatorTok\StringTok{ }
\StringTok{  }\NormalTok{dplyr}\OperatorTok{::}\KeywordTok{filter}\NormalTok{(Y2 }\OperatorTok{<=}\StringTok{ }\FloatTok{0.5} \OperatorTok{&}\StringTok{ }\NormalTok{Y2 }\OperatorTok{!=}\StringTok{ }\DecValTok{0}\NormalTok{) }\OperatorTok\StringTok{ }
\StringTok{  }\NormalTok{dplyr}\OperatorTok{::}\KeywordTok{filter}\NormalTok{(from }\OperatorTok{!=}\StringTok{ }\NormalTok{to }\OperatorTok{&}\StringTok{ }\NormalTok{from }\OperatorTok{>}\StringTok{ }\DecValTok{16} \OperatorTok{&}\StringTok{ }\NormalTok{from }\OperatorTok{<=}\StringTok{ }\DecValTok{21}\NormalTok{)}

\NormalTok{Y2.sl.AMP <-}\StringTok{ }\NormalTok{Y2.sl }\OperatorTok
\StringTok{  }\NormalTok{dplyr}\OperatorTok{::}\KeywordTok{left_join}\NormalTok{(., sites, }\DataTypeTok{by =} \KeywordTok{c}\NormalTok{(}\StringTok{"to"}\NormalTok{ =}\StringTok{ "site"}\NormalTok{)) }\OperatorTok\StringTok{ }
\StringTok{  }\NormalTok{dplyr}\OperatorTok{::}\KeywordTok{left_join}\NormalTok{(., sites, }\DataTypeTok{by =} \KeywordTok{c}\NormalTok{(}\StringTok{"from"}\NormalTok{ =}\StringTok{ "site"}\NormalTok{)) }\OperatorTok\StringTok{ }
\StringTok{  }\NormalTok{dplyr}\OperatorTok{::}\KeywordTok{filter}\NormalTok{(Y2 }\OperatorTok{<=}\StringTok{ }\FloatTok{0.5} \OperatorTok{&}\StringTok{ }\NormalTok{Y2 }\OperatorTok{!=}\StringTok{ }\DecValTok{0}\NormalTok{) }\OperatorTok\StringTok{ }
\StringTok{  }\NormalTok{dplyr}\OperatorTok{::}\KeywordTok{filter}\NormalTok{(from }\OperatorTok{!=}\StringTok{ }\NormalTok{to }\OperatorTok{&}\StringTok{ }\NormalTok{from }\OperatorTok{>}\StringTok{ }\DecValTok{21} \OperatorTok{&}\StringTok{ }\NormalTok{from }\OperatorTok{<=}\StringTok{ }\DecValTok{41}\NormalTok{)}

\NormalTok{Y2.sl.ECTZ <-}\StringTok{ }\NormalTok{Y2.sl }\OperatorTok
\StringTok{  }\NormalTok{dplyr}\OperatorTok{::}\KeywordTok{left_join}\NormalTok{(., sites, }\DataTypeTok{by =} \KeywordTok{c}\NormalTok{(}\StringTok{"to"}\NormalTok{ =}\StringTok{ "site"}\NormalTok{)) }\OperatorTok\StringTok{ }
\StringTok{  }\NormalTok{dplyr}\OperatorTok{::}\KeywordTok{left_join}\NormalTok{(., sites, }\DataTypeTok{by =} \KeywordTok{c}\NormalTok{(}\StringTok{"from"}\NormalTok{ =}\StringTok{ "site"}\NormalTok{)) }\OperatorTok\StringTok{ }
\StringTok{  }\NormalTok{dplyr}\OperatorTok{::}\KeywordTok{filter}\NormalTok{(Y2 }\OperatorTok{<=}\StringTok{ }\FloatTok{0.5} \OperatorTok{&}\StringTok{ }\NormalTok{Y2 }\OperatorTok{!=}\StringTok{ }\DecValTok{0}\NormalTok{) }\OperatorTok\StringTok{ }
\StringTok{  }\NormalTok{dplyr}\OperatorTok{::}\KeywordTok{filter}\NormalTok{(from }\OperatorTok{!=}\StringTok{ }\NormalTok{to }\OperatorTok{&}\StringTok{ }\NormalTok{from }\OperatorTok{>}\StringTok{ }\DecValTok{41}\NormalTok{)}

\CommentTok{# Load coastline}
\KeywordTok{load}\NormalTok{(}\StringTok{"data/coast.RData"}\NormalTok{)}

\NormalTok{sa_lats <-}\StringTok{ }\KeywordTok{c}\NormalTok{(}\OperatorTok{-}\DecValTok{38}\NormalTok{, }\OperatorTok{-}\DecValTok{26}\NormalTok{); sa_lons <-}\StringTok{ }\KeywordTok{c}\NormalTok{(}\DecValTok{14}\NormalTok{, }\DecValTok{34}\NormalTok{)}

\NormalTok{net.plot.Y1 <-}\StringTok{ }\ControlFlowTok{function}\NormalTok{(}\DataTypeTok{dissim =} \OtherTok{NULL}\NormalTok{, }\DataTypeTok{title =} \OtherTok{NULL}\NormalTok{, }\DataTypeTok{col.seq =} \OtherTok{NULL}\NormalTok{) \{}
  \KeywordTok{ggplot}\NormalTok{(dissim, }\KeywordTok{aes}\NormalTok{(lon.x, lat.x)) }\OperatorTok{+}
\StringTok{    }\KeywordTok{geom_polygon}\NormalTok{(}\DataTypeTok{data =}\NormalTok{ south_africa_coast, }
                 \KeywordTok{aes}\NormalTok{(}\DataTypeTok{x =}\NormalTok{ long, }\DataTypeTok{y =}\NormalTok{ lat, }\DataTypeTok{group =}\NormalTok{ group), }
                 \DataTypeTok{show.legend =} \OtherTok{FALSE}\NormalTok{, }\DataTypeTok{fill =} \StringTok{"#F9FAEC"}\NormalTok{) }\OperatorTok{+}
\StringTok{    }\KeywordTok{geom_curve}\NormalTok{(}\KeywordTok{aes}\NormalTok{(}\DataTypeTok{xend =}\NormalTok{ lon.y, }\DataTypeTok{yend =}\NormalTok{ lat.y, }\DataTypeTok{col =}\NormalTok{ Y1, }\DataTypeTok{alpha =}\NormalTok{ (}\DecValTok{1}\OperatorTok{-}\NormalTok{Y1)}\OperatorTok{-}\FloatTok{0.4}\NormalTok{),}
               \DataTypeTok{curvature =} \FloatTok{0.3}\NormalTok{) }\OperatorTok{+}\StringTok{ }
\StringTok{    }\KeywordTok{geom_point}\NormalTok{(}\DataTypeTok{data =}\NormalTok{ sites, }\KeywordTok{aes}\NormalTok{(}\DataTypeTok{x =}\NormalTok{ lon, }\DataTypeTok{y =}\NormalTok{ lat, }\DataTypeTok{fill =}\NormalTok{ bioreg}\OperatorTok{$}\NormalTok{bolton), }
               \DataTypeTok{col =} \StringTok{"black"}\NormalTok{, }\DataTypeTok{shape =} \DecValTok{21}\NormalTok{) }\OperatorTok{+}
\StringTok{    }\KeywordTok{scale_fill_manual}\NormalTok{(}\DataTypeTok{breaks =} \KeywordTok{c}\NormalTok{(}\StringTok{"AMP"}\NormalTok{, }\StringTok{"B-ATZ"}\NormalTok{, }\StringTok{"BMP"}\NormalTok{, }\StringTok{"ECTZ"}\NormalTok{),}
                      \DataTypeTok{values =}\NormalTok{ col.seq, }\DataTypeTok{name =} \StringTok{"bioregion"}\NormalTok{, }\DataTypeTok{guide =} \OtherTok{FALSE}\NormalTok{) }\OperatorTok{+}
\StringTok{    }\KeywordTok{scale_colour_gradient}\NormalTok{(}\DataTypeTok{name =} \KeywordTok{expression}\NormalTok{(}\KeywordTok{paste}\NormalTok{(beta[sim])), }
                          \DataTypeTok{low =} \StringTok{"black"}\NormalTok{, }\DataTypeTok{high =} \StringTok{"red"}\NormalTok{) }\OperatorTok{+}
\StringTok{    }\KeywordTok{coord_fixed}\NormalTok{(}\DataTypeTok{ratio =} \DecValTok{1}\NormalTok{, }\DataTypeTok{expand =} \OtherTok{TRUE}\NormalTok{) }\OperatorTok{+}
\StringTok{    }\KeywordTok{scale_x_continuous}\NormalTok{(}\DataTypeTok{breaks =} \KeywordTok{seq}\NormalTok{(}\DecValTok{15}\NormalTok{, }\DecValTok{35}\NormalTok{, }\DecValTok{5}\NormalTok{),}
                       \DataTypeTok{labels =}\NormalTok{ scales}\OperatorTok{::}\KeywordTok{unit_format}\NormalTok{(}\StringTok{"°E"}\NormalTok{, }\DataTypeTok{sep =} \StringTok{""}\NormalTok{)) }\OperatorTok{+}
\StringTok{    }\KeywordTok{scale_y_continuous}\NormalTok{(}\DataTypeTok{breaks =} \KeywordTok{seq}\NormalTok{(}\OperatorTok{-}\DecValTok{35}\NormalTok{, }\OperatorTok{-}\DecValTok{25}\NormalTok{, }\DecValTok{5}\NormalTok{),}
                       \DataTypeTok{labels =} \KeywordTok{c}\NormalTok{(}\StringTok{"35°S"}\NormalTok{, }\StringTok{"30°S"}\NormalTok{, }\StringTok{"25°S"}\NormalTok{)) }\OperatorTok{+}
\StringTok{    }\KeywordTok{scale_alpha_continuous}\NormalTok{(}\DataTypeTok{guide =} \OtherTok{FALSE}\NormalTok{) }\OperatorTok{+}
\StringTok{    }\KeywordTok{theme_grey}\NormalTok{() }\OperatorTok{+}\StringTok{ }\KeywordTok{xlab}\NormalTok{(}\OtherTok{NULL}\NormalTok{) }\OperatorTok{+}\StringTok{ }\KeywordTok{ylab}\NormalTok{(}\OtherTok{NULL}\NormalTok{) }\OperatorTok{+}
\StringTok{    }\KeywordTok{theme}\NormalTok{(}\DataTypeTok{panel.grid.minor =} \KeywordTok{element_blank}\NormalTok{()) }\OperatorTok{+}
\StringTok{    }\KeywordTok{ggtitle}\NormalTok{(title)\}}

\NormalTok{a <-}\StringTok{ }\KeywordTok{net.plot.Y1}\NormalTok{(Y1.sl.BMP, }\StringTok{"Benguela Marine Province"}\NormalTok{, }
                 \DataTypeTok{col.seq =} \KeywordTok{c}\NormalTok{(}\StringTok{"black"}\NormalTok{, }\StringTok{"black"}\NormalTok{, }\StringTok{"white"}\NormalTok{, }\StringTok{"black"}\NormalTok{)) }\OperatorTok{+}\StringTok{ }\CommentTok{# alphabetical}
\StringTok{  }\KeywordTok{theme}\NormalTok{(}\DataTypeTok{legend.direction =} \StringTok{"horizontal"}\NormalTok{,}
        \DataTypeTok{legend.position =} \KeywordTok{c}\NormalTok{(}\DataTypeTok{x =} \FloatTok{0.5}\NormalTok{, }\DataTypeTok{y =} \FloatTok{0.8}\NormalTok{),}
        \DataTypeTok{legend.key.height =} \KeywordTok{unit}\NormalTok{(}\FloatTok{0.3}\NormalTok{, }\StringTok{"cm"}\NormalTok{)) }
\NormalTok{b <-}\StringTok{ }\KeywordTok{net.plot.Y1}\NormalTok{(Y1.sl.BATZ, }\StringTok{"Benguela-Agulhas Transition Zone"}\NormalTok{, }
                 \DataTypeTok{col.seq =} \KeywordTok{c}\NormalTok{(}\StringTok{"black"}\NormalTok{, }\StringTok{"white"}\NormalTok{, }\StringTok{"black"}\NormalTok{, }\StringTok{"black"}\NormalTok{)) }\OperatorTok{+}
\StringTok{  }\KeywordTok{theme}\NormalTok{(}\DataTypeTok{legend.position =} \StringTok{"none"}\NormalTok{)}
\NormalTok{c <-}\StringTok{ }\KeywordTok{net.plot.Y1}\NormalTok{(Y1.sl.AMP, }\StringTok{"Agulhas Marine Province"}\NormalTok{, }
                 \DataTypeTok{col.seq =} \KeywordTok{c}\NormalTok{(}\StringTok{"white"}\NormalTok{, }\StringTok{"black"}\NormalTok{, }\StringTok{"black"}\NormalTok{, }\StringTok{"black"}\NormalTok{)) }\OperatorTok{+}
\StringTok{  }\KeywordTok{theme}\NormalTok{(}\DataTypeTok{legend.position =} \StringTok{"none"}\NormalTok{)}
\NormalTok{d <-}\StringTok{ }\KeywordTok{net.plot.Y1}\NormalTok{(Y1.sl.ECTZ, }\StringTok{"East Coast Transition Zone"}\NormalTok{, }
                 \DataTypeTok{col.seq =} \KeywordTok{c}\NormalTok{(}\StringTok{"black"}\NormalTok{, }\StringTok{"black"}\NormalTok{, }\StringTok{"black"}\NormalTok{, }\StringTok{"white"}\NormalTok{)) }\OperatorTok{+}
\StringTok{  }\KeywordTok{theme}\NormalTok{(}\DataTypeTok{legend.position =} \StringTok{"none"}\NormalTok{)}

\KeywordTok{pdf}\NormalTok{(}\StringTok{"Fig4.pdf"}\NormalTok{, }\DataTypeTok{width =} \DecValTok{7}\NormalTok{, }\DataTypeTok{height =} \FloatTok{3.8}\NormalTok{)}
\NormalTok{grid}\OperatorTok{::}\KeywordTok{grid.newpage}\NormalTok{()}
\NormalTok{grid}\OperatorTok{::}\KeywordTok{pushViewport}\NormalTok{(grid}\OperatorTok{::}\KeywordTok{viewport}\NormalTok{(}\DataTypeTok{layout =}\NormalTok{ grid}\OperatorTok{::}\KeywordTok{grid.layout}\NormalTok{(}\DecValTok{2}\NormalTok{,}\DecValTok{2}\NormalTok{)))}
\NormalTok{vplayout <-}\StringTok{ }\ControlFlowTok{function}\NormalTok{(x, y) grid}\OperatorTok{::}\KeywordTok{viewport}\NormalTok{(}\DataTypeTok{layout.pos.row =}\NormalTok{ x, }\DataTypeTok{layout.pos.col =}\NormalTok{ y)}
\KeywordTok{print}\NormalTok{(a, }\DataTypeTok{vp =} \KeywordTok{vplayout}\NormalTok{(}\DecValTok{1}\NormalTok{,}\DecValTok{1}\NormalTok{))}
\KeywordTok{print}\NormalTok{(b, }\DataTypeTok{vp =} \KeywordTok{vplayout}\NormalTok{(}\DecValTok{1}\NormalTok{,}\DecValTok{2}\NormalTok{))}
\KeywordTok{print}\NormalTok{(c, }\DataTypeTok{vp =} \KeywordTok{vplayout}\NormalTok{(}\DecValTok{2}\NormalTok{,}\DecValTok{1}\NormalTok{))}
\KeywordTok{print}\NormalTok{(d, }\DataTypeTok{vp =} \KeywordTok{vplayout}\NormalTok{(}\DecValTok{2}\NormalTok{,}\DecValTok{2}\NormalTok{))}
\KeywordTok{dev.off}\NormalTok{()}
\end{Highlighting}
\end{Shaded}

And that's it, folks. You'll notice that I haven't reproduced Fig. 5
here. I'll leave that up to you\ldots{} or ask me and I'll send the
code. All of the matrices have (mostly) been calculated above and they
can be used together with some \textbf{dplyr} and \textbf{ggplot2}
know-how in the creation of that graph.

Legalise seaweed!

\section*{References}

\hypertarget{refs}{}
\hypertarget{ref-Anderson2013}{}
Anderson, M. J., Tolimieri, N., and Millar, R. B. (2013). Beta diversity
of demersal fish assemblages in the North-Eastern Pacific: interactions
of latitude and depth. \emph{PLOS ONE} 8, e57918.

\hypertarget{ref-Baselga2010}{}
Baselga, A. (2010). Partitioning the turnover and nestedness components
of beta diversity. \emph{Global Ecology and Biogeography} 19, 134--143.

\hypertarget{ref-Baselga2013}{}
Baselga, A., Orme, D., Villeger, S., Bortoli, J. D., and Leprieur, F.
(2013). \emph{betapart: Partitioning beta diversity into turnover and
nestedness components}. Available at:
\url{http://CRAN.R-project.org/package=betapart}.

\hypertarget{ref-Blanchet2008}{}
Blanchet, F. G., Legendre, P., and Borcard, D. (2008). Forward selection
of explanatory variables. \emph{Ecology} 89, 2623--2632.

\hypertarget{ref-Bolton1986}{}
Bolton, J. J. (1986). Marine phytogeography of the Benguela upwelling
region on the west coast of southern Africa: A temperature dependent
approach. \emph{Botanica Marina} 29, 251--256.

\hypertarget{ref-Bolton2004}{}
Bolton, J. J., and Anderson, R. J. (2004). ``Marine Vegetation,'' in
\emph{Vegetation of southern africa}, eds. R. M. Cowling, D. M.
Richardson, and S. M. Pierce (Cambridge University Press), 348--370.

\hypertarget{ref-Bolton2002}{}
Bolton, J. J., and Stegenga, H. (2002). Seaweed species diversity in
South Africa. \emph{South African Journal of Marine Science} 24, 9--18.

\hypertarget{ref-Borcard2011}{}
Borcard, D., Gillet, F., and Legendre, P. (2011). \emph{Numerical
Ecology with R}. Springer New York Available at:
\url{https://books.google.co.za/books?id=dtQNxsH4Y2wC}.

\hypertarget{ref-Davidar2007}{}
Davidar, P., Rajagopal, B., Mohandass, D., Puyravaud, J.-P., Condit, R.,
Wright, S., et al. (2007). The effect of climatic gradients, topographic
variation and species traits on the beta diversity of rain forest trees.
\emph{Global Ecology and Biogeography} 16, 510--518.

\hypertarget{ref-DeClerck2005}{}
De Clerck, O., Bolton, J. J., Anderson, R. J., and Coppejans, E. (2005).
Guide to the seaweeds of KwaZulu-Natal. \emph{Scripta Botanica Belgica}
33, 294 pp.

\hypertarget{ref-Dray2006}{}
Dray, S., Legendre, P., and Peres-Neto, P. R. (2006). Spatial modelling:
a comprehensive framework for principal coordinate analysis of neighbour
matrices (PCNM). \emph{Ecological Modelling} 196, 483--493.

\hypertarget{ref-Dray2012a}{}
Dray, S., Pélissier, R., Couteron, P., Fortin, M. J., Legendre, P.,
Peres-Neto, P. R., et al. (2012). Community ecology in the age of
multivariate multiscale spatial analysis. \emph{Ecological Monographs}
82, 257--275.

\hypertarget{ref-Nekola1999}{}
Nekola, J. C., White, P. S., Carolina, N., Nekola, C., and Curriculum,
P. S. W. (1999). The distance decay of similarity in biogeography and
ecology. \emph{Journal of Biogeography} 26, 867--878.

\hypertarget{ref-PeresNeto2010}{}
Peres-Neto, P. R., and Legendre, P. (2010). Estimating and controlling
for spatial structure in the study of ecological communities.
\emph{Global Ecology and Biogeography} 19, 174--184.

\hypertarget{ref-Peres-Neto2006}{}
Peres-Neto, P. R., Legendre, P., Dray, S., and Borcard, D. (2006).
Variation partitioning of species data matrices: estimation and
comparison of fractions. \emph{Ecology} 87, 2614--2625.

\hypertarget{ref-Sauer1988}{}
Sauer, J. D. (1988). \emph{Plant migration: The dynamics of geographic
patterning in seed plant species}. University of California Press.

\hypertarget{ref-Smit2013}{}
Smit, A. J., Roberts, M., Anderson, R. J., Dufois, F., Dudley, S. F. J.,
Bornman, T. G., et al. (2013). A coastal seawater temperature dataset
for biogeographical studies: large biases between \emph{in situ} and
remotely-sensed data sets around the coast of South Africa. \emph{PLOS
ONE} 8, e81944.

\hypertarget{ref-Stegenga1997}{}
Stegenga, H., Bolton, J. J., and Anderson, R. J. (1997). Seaweeds of the
South African west coast. \emph{Contributions of the Bolus Herbarium}
18, 3--637.

\hypertarget{ref-Wernberg2013}{}
Wernberg, T., Thomsen, M. S., Connell, S. D., Russell, B. D., Waters, J.
M., Zuccarello, G. C., et al. (2013). The footprint of continental-scale
ocean currents on the biogeography of seaweeds. \emph{PLOS ONE} 8,
e80168.

\end{document}
